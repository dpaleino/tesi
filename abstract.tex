\documentclass[a4paper,10pt]{article}
\usepackage{fontspec}
\usepackage[italian]{babel}

\title{Valutazione di analisi cefalometriche tradizionali vs. proporzionali: effetti sulla diagnosi}
\author{
David Paleino\\
\begingroup
\footnotesize
Matr. 0506251
\endgroup
}
\date{}

\begin{document}

\maketitle

\begin{abstract}
La cefalometria è uno strumento incomparabile di studio, di diagnosi, di pianificazione di trattamento e di valutazione della crescita, con o senza trattamento. Essa è principalmente usata in Ortodonzia, ma anche in Chirurgia Maxillo-Facciale, in Pedodonzia, in Protesi o in Chirurgia Plastica.

Il concetto di normalità in biometria è una nozione difficile da definire, ma che potrebbe definirsi come appartenente alla norma statistica. La norma ideale corrisponde ai valori medi della media aritmetica. L’intervallo di dispersione della normalità è abbastanza vasto e raggruppa quasi il 70\% della popolazione. È per questo motivo che, nel corso degli anni, alcuni Autori hanno ideato tecniche cefalometriche atte ad individualizzare il processo diagnostico, non affidandosi più a medie statistiche di popolazione, ma sfruttando le proporzioni esistenti tra le varie componenti dell'apparato stomatognatico.

Questo studio si propone di presentare alcune tecniche di analisi cefalometrica, di seguito definite \emph{proporzionali}, e di valutarne il differente impatto in termini diagnostici rispetto a quelle tecniche che si basano su valori ideali comuni per tutti, \emph{statisticamente rilevati}.

In particolare, verranno prese in esame, a livello teorico, le seguenti tecniche:
\begin{itemize}
\item analisi statistiche
\begin{itemize}
\item Downs
\item Steiner (secondo Giannì)
\item Ricketts
\end{itemize}
\item analisi proporzionali
\begin{itemize}
\item analisi delle controparti di Enlow
\item analisi proporzionale di Coben
\item analisi architetturale di Delaire
\item la rosa dei venti di Sassouni
\end{itemize}
\end{itemize}

Il disegno sperimentale prevede invece la valutazione di 29 teleradiografie in proiezione latero-laterale con craniostato, e di ognuna verrà effettuata un'analisi di tipo statistico (Giannì), e due di tipo proporzionale (Enlow, Coben). Queste verranno quindi comparate in termini diagnostici (classe scheletrica e altre valutazioni). Per quanto riguarda la classe scheletrica, verranno presi in considerazione anche l'indice Wits e gli appropriati valori dell'analisi di Ricketts.

Scopo della tesi è quindi di valutare l'eventuale concordanza o discordanza tra le diverse analisi cefalometriche considerate.
\end{abstract}
\end{document}
