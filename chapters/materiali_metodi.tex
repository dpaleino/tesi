\chapter{Materiali e metodi}
Questo studio è stato eseguito su un campione totale è di 29 pazienti, con un'età media di $11,29 \pm 3,10$ anni, composto da 10 pazienti di sesso maschile e 19 pazienti di sesso femminile. I pazienti erano tutti afferenti al reporto di Ortognatodonzia del Dipartimento di Scienze Stomatologiche ``Giuseppe Messina'' dell'Azienda Ospedaliera Universitaria Policlinico ``Paolo Giaccone'' di Palermo. I nomi dei pazienti sono stati codificati in modo da garantirne l'anonimato.

I tracciati cefalometrici sono stati effettuati manualmente da un singolo operatore. 

Per effettuare l'analisi dei dati sono stati utilizzati lo \emph{scostamento medio}, il \emph{$\chi^2$ di Pearson} e la \emph{$\kappa$ di Fleiss}.

\section{Scostamento medio}
\label{scostamento_medio}
Questo strumento permette di valutare il grado di \emph{dispersione} dei valori attorno alla loro media. Questo è molto utile nella componente verticale dell'analisi delle controparti di Enlow: essendo un'analisi proporzionale non è possibile riferire le misurazioni ad un valore preciso, ma possiamo confrontarle tra di loro.

Lo \emph{scostamento medio} è definito come \emph{la media degli scarti assoluti dei valori osservati rispetto al loro valore medio}:
\begin{equation}
\label{eq:scostamento_medio}
\varepsilon = \frac{1}{n}\sum_{i=1}^n{\left(\left|x_i - \frac{1}{n}\sum_{k=1}^n x_k\right|\right)}
\end{equation}

Uno scostamento elevato significherà che i valori sono lontani tra di loro (e quindi falliranno il requisito imposto da Enlow che abbiano ``minime variazioni''); un numero basso significherà invece che i valori differiscono in misura minore; un valore pari a 0 indica che i valori sono uguali tra di loro.

\section{$\kappa$ di Fleiss}
Questo metodo permette di valutare quanto la concordanza delle valutazioni di due o più osservatori, su un determinato campione, sia dovuto al caso, e stimare quindi la concordanza depurata dal caso stesso.
\begin{equation}
\label{eq:kappa_fleiss}
\kappa = \frac{\overline{P} - \overline{P}_e}{1 - \overline{P}_e}
\end{equation}

Il denominatore $1 - \overline{P}_e$ espime il grado di concordanza ottenibile al di fuori del caso; il numeratore $\overline{P} - \overline{P}_e$ esprime la concordanza ottenuta al di fuori del caso. Se le analisi fossero in completo accordo, allora $\kappa = 1$, all'opposto, se non ci fosse accordo, oltre a quello dovuto al caso, si avrebbe $\kappa \leq 0$.

Sia $N$ il numero totale del campione, $n$ il numero degli osservatori e $k$ il numero di categorie assegnate ai vari componenti del campione. I soggetti vengono numerati da $i = 1, \ldots N$, e le categorie da $j = 1, \ldots k$. Sia $n_{ij}$ il numero di osservatori che hanno assegnato il soggetto $i$ alla categoria $j$.

Prima è necessario calcolare $p_i$, che è la proporzione delle valutazioni in quella categoria rispetto a tutte le categorie:
\begin{equation*}
p_j = \frac{1}{Nn}\sum_{i=1}^N n_{ij}
\end{equation*}
Questo viene usato per calcolare $\overline{P}_e$:
\begin{equation}
\overline{P}_e = \sum_{j=1}^k p^2_j
\end{equation}
$P_i$ rappresenta il grado di concordanza tra gli osservatori per il soggetto $i$-esimo.
\begin{align*}
P_i &= \frac{1}{n(n-1)}\sum_{j=1}^k n_{ij}(n_{ij}-1) \\
&= \frac{1}{n(n-1)}\sum_{j=1}^k (n^2_{ij}-n_{ij}) \\
&= \frac{1}{n(n-1)}[(\sum_{j=1}^k n^2_{ij}) - n]
\end{align*}
$\overline{P}$, utilizzato nell'equazione principale, è la media dei $P_i$.
\begin{align*}
\overline{P} &= \frac{1}{N}\sum_{i=1}^N P_i \\
&= \frac{1}{Nn(n-1)}(\sum_{i=1}^N\sum_{j=1}^k n^2_{ij} - Nn)
\end{align*}
Per semplificare, in questa tesi i confronti sono stati fatti a coppie, quindi il numero di osservatori è sempre $2$. Con questa semplificazione, le equazioni diventano:
\begin{equation}
P_i = \frac{1}{2}[(\sum_{j=1}^k n^2_{ij}) - 2]
\end{equation}
\begin{equation}
\overline{P} = \frac{1}{2N}(\sum_{i=1}^N\sum_{j=1}^k n^2_{ij} - 2N)
\end{equation}
Il valore di $\kappa$ viene valutato in base ad una tabella (\vref*{tab:significato_kappa}), proposta da Landis e Koch\footcite{Landis1977}.

\begin{table}[ht]
\centering
\caption{Significato dei valori di $\kappa$}
\label{tab:significato_kappa}
\begin{tabular}{ll}
\toprule
\multicolumn{1}{c}{\textbf{$\kappa$}} & \multicolumn{1}{c}{significato} \\
\midrule
$< 0$ & scarso accordo \\
$0,01 - 0,20$ & leggero accordo \\
$0,21 - 0,40$ & buon accordo \\
$0,41 - 0,61$ & moderato accordo \\
$0,61 - 0,81$ & sostanziale accordo \\
$0,81 - < 1,00$ & accordo quasi perfetto \\
$1,00$ & accordo perfetto \\
\bottomrule
\end{tabular}
\end{table}