\chapter{Conclusioni}

- necessità simbiosi tra proporzionale e statistica\\
- proporzionale indica l'opportunità di fare terapia, ma non che tipo di terapia (non si capisce di chi è la colpa dello squilibrio); non legge le bi-{pro,re}trusioni\\
- statistica inaffidabile perché riferisce a standard validi per tutti (non legge le compensazioni? VERIFICARE)\\
- le proporzionali necessitano di un ragionamento sull'insieme delle misure, non sulla singola\\
- proporzionali meno indicazioni terapeutiche\\
- non vengono considerate le differenze tra razze, o tra le fasce d'età\\
- proporzionali per ricerca, statistiche per terapia (esempio Giannì)\\
- notevole discrepanza di dati, ridimensiona il valore diagnostico generale dell'analisi cefalometrica, che dev'essere combinata con la valutazione di altri parametri relativi ai valori dentali e ai tessuti molli