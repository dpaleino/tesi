\chapter{Conclusioni}

È evidente come i risultati ottenuti da questo studio mostrino un'eterogeneità di valutazioni in base al tipo di analisi cefalometrica utilizzata.

L'utilità diagnostica della cefalometria ne esce ridimensionata: essa resta uno strumento utile per impostare un piano di trattamento, ma non è un valore assoluto, in quanto dev'essere integrata con la valutazione di altri parametri (dentali e dei tessuti molli -- non considerati in questo lavoro), e, più in generale, con l'``esperienza clinica'', che può individuare quanto un'indicazione data dall'analisi cefalometrica possa essere corretta o meno.

I risultati ottenuti suggeriscono che le analisi proporzionali siano state ideate a scopi di ricerca, o come controllo longitudinale dei pazienti: questo è l'intento dichiarato di Enlow (che introduce la propria analisi parlando dei \emph{cambiamenti regionali}, ossia dei vettori di crescita delle varie porzioni del cranio), e presumibilmente lo è anche di Coben, che nei suoi lavori non propone mai indicazioni terapeutiche, e viene usato anche da altri autori come \emph{verifica} degli effetti di un trattamento. Questo è in netto contrasto con le analisi riferite a valori statistici, che generalmente forniscono indicazioni terapeutiche: lo stesso Giannì, per esempio, riprendendo l'analisi di Steiner e arricchendola, propone interventi terapeutici di correzione della classe scheletrica in base all'equilibrio verticale anteriore (normo-verti-bite, open-bite e deep-bite) e alla presunta sede del dismorfismo.

È necessario però tenere in considerazione anche i limiti delle analisi riferite a valori statistici: queste infatti includono nel range di normalità solamente il $68\%$ della popolazione (per la definizione di \emph{deviazione standard}). La maggior parte delle analisi riferite a valori statistici è stata inoltre creata sullo standard caucasico -- e questo non è un dato trascurabile in una società sempre più multietnica -- e non distinguono tra le varie fasi dello sviluppo (e quindi fasce d'età, che alcuni autori però considerano, per esempio Ricketts).

%È quindi necessaria una simbiosi tra le analisi di tipo proporzionale e di tipo statistico: le prime riescono più efficacemente ad indicare la necessità di un'eventuale terapia, che va riconsiderata alla luce di un'analisi del secondo tipo. Per esempio, l'analisi di Enlow riesce a distinguere una disarmonia tra la mascella e la mandibola, ma non riesce a distinguere dove si trovi il dismorfismo, né riesce a leggere le biprotrusioni o le biretrusioni. Per questo tipo di valutazioni è necessario effettuare un'altra analisi.

Un ulteriore limite delle analisi proporzionali è che il singolo valore ha poco significato, se non si osserva come è stato ottenuto. Proprio per la loro natura di ``proporzionali'', ogni valore viene riferito ad un altro: uno squilibrio può quindi essere insito nel parametro valutato, o nella misura di riferimento. In Coben, per esempio, tutte le misure vengono riferite alla base cranica: se questa è molto lunga, o molto corta, tutte le misure dipendenti da questa verranno falsate. È quindi necessario non concentrarsi sul singolo valore, ma è più proficuo considerare l'analisi nel suo insieme.

%- necessità simbiosi tra proporzionale e statistica\\
%- proporzionale indica l'opportunità di fare terapia, ma non che tipo di terapia (non si capisce di chi è la colpa dello squilibrio); non legge le bi-{pro,re}trusioni\\
%- statistica inaffidabile perché riferisce a standard validi per tutti (non legge le compensazioni? VERIFICARE)\\
%- le proporzionali necessitano di un ragionamento sull'insieme delle misure, non sulla singola\\
%- proporzionali meno indicazioni terapeutiche\\
%- non vengono considerate le differenze tra razze, o tra le fasce d'età\\
%- proporzionali per ricerca, statistiche per terapia (esempio Giannì)\\
%- notevole discrepanza di dati, ridimensiona il valore diagnostico generale dell'analisi cefalometrica, che dev'essere combinata con la valutazione di altri parametri relativi ai valori dentali e ai tessuti molli