\chapter{Discussione}
Questo studio prevede la valutazione di alcuni parametri utilizzando diverse tecniche cefalometriche, e analizzati in senso statistico.
%Verranno di seguito analizzati i risultati delle misurazioni effettuate su due piani:  antero-posteriore (classe scheletrica) e verticale (rotazione del piano occlusale).

\section{Classe scheletrica}
Per valutare il rapporto antero-posteriore tra il mascellare superiore e la mandibola (\emph{classe scheletrica}) sono state prese in considerazione cinque analisi cefalometriche: tre riferite a valori statistici (Giannì, Wits, Ricketts), e due riferite a valori proporzionali (Enlow, Coben).

Per quanto riguarda l'analisi di Giannì, è stato valutato l'angolo \angolo{ANB}; l'indice di Wits è notoriamente la distanza millimetrica \piano{$A_0$}{$B_0$} sul piano occlusale; in Ricketts viene valutato l'angolo facciale (\piano{N}{Pog} su \punto{FH}) e la convessità facciale (distanza ortogonale di \punto{A} da \piano{N}{Pog}). Nell'analisi delle controparti di Enlow è stata valutata la discrepanza tra le controparti mascellare basale (\piano{A}{PM}) e mandibolare basale (\piano{B$\perp$REF}{ARA}). In Coben sono state fatte due valutazioni, differenti tra loro nel riferimento mandibolare: \piano{Ba}{Pog} per una, e \piano{Ba}{B} per l'altra, entrambe rapportate a \piano{Ba}{A}. Queste verranno in seguito denominate ``Coben \punto{Pog}'' e ``Coben \punto{B}'', rispettivamente.

Dato che non esiste un \emph{gold standard} per il rilevamento della classe scheletrica, sono stati eseguiti tre \emph{test $\chi^2$ di Pearson}, considerando \emph{standard} di volta in volta le tre analisi riferite a valori statistici. L'ipotesi nulla ($H_0$) è, in tutti e tre i casi, che \emph{non esiste alcuna differenza tra le analisi considerate}.

La tabella delle frequenze delle classi scheletriche (\ref{tab:classe_scheletrica_frequenze}) mostra già un'evidente discordanza numerica nel rilevamento delle III classi tra Giannì e le altre analisi.

\begin{table}
\centering
\caption{Frequenza delle classi scheletriche}
\label{tab:classe_scheletrica_frequenze}
\begin{tabular}{>{\bfseries}c*{6}{c}}
\toprule
Classe & Enlow & Giannì & Wits & Ricketts & Coben \punto{Pog} & Coben \punto{B} \\
\midrule
I & 11 & 16 & 13 & 18 & 11 & 9 \\
II & 11 & 12 & 8 & 8 & 12 & 13 \\
III & 7 & 1 & 8 & 3 & 6 & 7 \\
\bottomrule
\end{tabular}
\end{table}

Calcolando il $\chi^2$ con grado di libertà 2, e il valore $p$ (vedi paragrafo \ref{sec:chi-square} a pagina \pageref{sec:chi-square}) si ottengono le tabelle \ref{tab:classe_scheletrica_chi-quadro} e \ref{tab:classe_scheletrica_p-value} (in evidenza i valori di $p$ statisticamente significativi).

\begin{table}
\centering
\caption{Test del $\chi^2$ sulle classi scheletriche}
\label{tab:classe_scheletrica_chi-quadro}
\begin{tabular}{>{\bfseries}c*{6}{D{,}{,}{2.2}}}
\toprule
 & \multicolumn{1}{c}{Enlow} & \multicolumn{1}{c}{Giannì} & \multicolumn{1}{c}{Wits} & \multicolumn{1}{c}{Ricketts} & \multicolumn{1}{c}{Coben \punto{Pog}} & \multicolumn{1}{c}{Coben \punto{B}} \\
\midrule
Giannì & 37,65 & 0 & 50,90 & 5,58 & 26,56 & 39,15 \\
Wits & 1,56 & 8,82 & 0 & 5,05 & 2,81 & 4,48 \\
Ricketts & 9,18 & 3,56 & 9,72 & 0 & 7,72 & 12,96 \\
\bottomrule
\end{tabular}
\end{table}

\begin{table}
\centering
\caption{Valori $p$ nei test del $\chi^2$ sulle classi scheletriche}
\label{tab:classe_scheletrica_p-value}
\begin{tabular}{>{\bfseries}c*{6}{c}}
\toprule
 & \multicolumn{1}{c}{Enlow} & \multicolumn{1}{c}{Giannì} & \multicolumn{1}{c}{Wits} & \multicolumn{1}{c}{Ricketts} & \multicolumn{1}{c}{Coben \punto{Pog}} & \multicolumn{1}{c}{Coben \punto{B}} \\
\midrule
Giannì & \cellcolor[gray]{.8} $< 0,0001$ & $1$ & \cellcolor[gray]{.8} $< 0,0001$ & $0,0613$ & \cellcolor[gray]{.8} $< 0,0001$ & \cellcolor[gray]{.8} $< 0,0001$ \\
Wits & $0,4589$ & \cellcolor[gray]{.8} $0,0122$ & $1$ & $0,0801$ & $0,2457$ & $0,1064$ \\
Ricketts & \cellcolor[gray]{.8} $0,0101$ & $0,1690$ & \cellcolor[gray]{.8} $0,0077$ & $1$ & \cellcolor[gray]{.8} $0,0210$ & \cellcolor[gray]{.8} $0,0015$ \\
\bottomrule
\end{tabular}
\end{table}

Questi dati mettono in evidenza come, considerando \emph{standard} le analisi di Giannì e Ricketts, si ottengano valori $p$ statisticamente significativi quando paragonate con le analisi di Enlow, Wits e Coben. Questo significa che l'\emph{ipotesi nulla} è rigettata per queste tre analisi, ossia le differenze tra le diagnosi effettuate sono \emph{statisticamente significative}. Sembrerebbe che, invece, considerando come standard l'indice di Wits, le differenze siano meno significative, eccezion fatta per l'analisi di Giannì.

Quest'analisi, però, mette a confronto frequenze cumulative: è possibile che tutti i pazienti inseriti in una categoria secondo un'analisi, vengano inseriti in un'altra categoria secondo un'altra analisi. È necessario quindi analizzare la concordanza tra diagnosi per ogni singolo paziente. La tabella \vref{tab:classe_scheletrica_concordanza} mostra le percentuali di concordanza \emph{semplice}, intesa come il rapporto tra il numero di pazienti le cui diagnosi coincidono e il numero totale di soggetti.

\begin{table}
\centering
\caption{Concordanza semplice sulla classe scheletrica}
\label{tab:classe_scheletrica_concordanza}
\begin{tabular}{>{\bfseries}c*{5}{D{,}{,}{2.2}}}
\toprule
 & \multicolumn{1}{c}{\textbf{Enlow}} & \multicolumn{1}{c}{\textbf{Giannì}} & \multicolumn{1}{c}{\textbf{Wits}} & \multicolumn{1}{c}{\textbf{Ricketts}} & \multicolumn{1}{c}{\textbf{Coben \punto{Pog}}} \\
\midrule
Giannì & 58,62 & & 51,72 & 65,52 & 55,17 \\
Wits & 62,07 & 51,72 & & 65,52 & 48,28 \\
Ricketts & 68,97 & 65,52 & 65,52 & & 34,48 \\
Coben \punto{Pog} & 37,93 & 55,17 & 48,28 & 34,48 & \\
Coben \punto{B} & 44,83 & 55,17 & 55,17 & 41,38 & 93,1 \\
\bottomrule
\end{tabular}
\end{table}

Seppur non molto alte, molte percentuali superano il $50\%$. Questo tipo di concordanza, però, include quei casi in cui la coincidenza di diagnosi è dovuta esclusivamente al caso. Per eliminare questi casi, è possibile utilizzare la \emph{$\kappa$ di Fleiss}. I valori risultanti sono riportati in tabella \vref{tab:classe_scheletrica_kappa}.

\begin{table}
\centering
\caption{Concordanza depurata dal caso sulla classe scheletrica}
\label{tab:classe_scheletrica_kappa}
\begin{tabular}{>{\bfseries}c*{5}{D{,}{,}{2.2}}}
\toprule
 & \multicolumn{1}{c}{\textbf{Enlow}} & \multicolumn{1}{c}{\textbf{Giannì}} & \multicolumn{1}{c}{\textbf{Wits}} & \multicolumn{1}{c}{\textbf{Ricketts}} & \multicolumn{1}{c}{\textbf{Coben \punto{Pog}}} \\
\midrule
Giannì & 31,83 & & 20,47 & 35,27 & 24,98 \\
Wits & 42,05 & 20,47 & & 42,74 & 20,62 \\
Ricketts & 49,37 & 35,27 & 42,74 & & -7,93 \\
Coben \punto{Pog} & 4,31 & 24,98 & 20,62 & 34,48 &  \\
Coben \punto{B} & 15,33 & 26,44 & 31,89 & 5,83 & 89,31 \\
\bottomrule
\end{tabular}
\end{table}

Questi dati denotano un'importante discordanza tra le varie analisi considerate.

\section{Altezza facciale inferiore}
La valutazione dell'altezza facciale inferiore è stata effettuata prendendo in considerazione l'angolo della divergenza di Giannì (piano bispinale - piano mandibolare), e l'altezza facciale inferiore \piano{SNA}{Me} (considerata su \punto{BaV}) secondo Coben. Questo confronto è possibile perché un angolo intermascellare aumentato si riflette in un'aumentata altezza facciale inferiore. Come \emph{gold standard} è stata considerata l'analisi di Giannì. L'ipotesi nulla $H_0$ è che \emph{non esiste alcuna differenza tra le due analisi}.

La tabella \vref{tab:altezza_facciale_inferiore_frequenze} mostra le frequenze nella valutazione dell'altezza facciale inferiore per le due analisi. In questo caso, il test del $\chi^2$ non sarebbe applicabile, in quanto necessita di almeno 5 osservazioni per ogni categoria. Per poter ottenere ciò, è possibile unificare le osservazioni \emph{non-normali}: le frequenze così ottenute sono riportate in tabella \vref{tab:altezza_facciale_inferiore_frequenze_cumulative}.

\begin{table}
\centering
\begin{centeredcaptionlabel}{Frequenza delle modifiche all'altezza facciale inferiore}{tab:altezza_facciale_inferiore_frequenze}
% \label{}
\begin{tabular}{>{\bfseries}ccc}
\toprule
& \textbf{Giannì} & \textbf{Coben} \\
\midrule
Normale & 14 & 19 \\
Diminuita & 0 & 6 \\
Aumentata & 15 & 4 \\
\bottomrule
\end{tabular}
\end{centeredcaptionlabel}\hfill
\begin{centeredcaptionlabel}{Frequenza (cumulativa) delle modifiche all'altezza facciale inferiore}{tab:altezza_facciale_inferiore_frequenze_cumulative}
% \label{}
\begin{tabular}{>{\bfseries}ccc}
\toprule
& \textbf{Giannì} & \textbf{Coben} \\
\midrule
Normale & 14 & 19 \\
Non-normale & 15 & 10 \\
\bottomrule
\end{tabular}
\end{centeredcaptionlabel}
\end{table}

Il $\chi^2$ (un grado di libertà) è pari a $3,4524$, e il valore $p$ risultante è $0,632$. Questo è superiore al valore standard di $p < 0,05$, il che significa che non c'è alcuna differenza statisticamente significativa tra le due analisi. Essendo molto vicino al valore $p$ limite, però, è possibile affermare che esiste una \emph{significatività pratica}.

Per poter analizzare le frequenze non cumulative, anche in questo caso è necessario valutare la concordanza per singolo paziente, e quindi ricorrere alla \emph{$\kappa$ di Fleiss}. La concordanza semplice è del $34,48\%$; la concordanza depurata dal caso, secondo Fleiss, è del $-17,36\%$. Questo significa che non solo le due analisi concordano poco, ma la concordanza esistente è generalmente dovuta al caso.

La differenza nei risultati è probabilmente dovuta al fatto che l'altezza facciale in Coben è una misura riferita ad un piano perpendicolare al piano di Francoforte: una diversa angolazione di quest'ultimo nei pazienti causa una diversa lettura di una stessa altezza facciale, generando le forti discrepanze rilevate nel corso di questo paragrafo.

\section{Rotazione del piano occlusale}
Per poter valutare la rotazione del piano occlusale è necessario considerare che l'analisi delle controparti di Enlow non fornisce dei valori di riferimento, per cui bisogna valutare il rapporto delle controparti verticali tra di loro. Per far questo, verrà utilizzato lo \emph{scostamento medio} descritto nel paragrafo \vref{scostamento_medio}. Oltre a questo però, bisogna prendere in considerazione il rapporto tra la componente verticale anteriore e le altre due controparti: se essa è maggiore, il piano occlusale sarà in post-rotazione; se è minore anche di una sola delle due, sarà in antero-rotazione. Per poter definire il tipo di rotazione del piano occlusale, è necessario scegliere un $\varepsilon$ appropriato al di sopra del quale è possibile definire \emph{in squilibrio} le controparti verticali di Enlow, e perciò parlare di una eccessiva inclinazione nell'uno o nell'altro senso.

Secondo Giannì, l'angolo cranio-occlusale definisce l'inclinazione del piano occlusale: il suo valore medio è $14 \pm 3°$; valori minori indicano ante-rotazione occlusale; valori superiori indicano post-rotazione.

Il grafico \vref{graph:inclinazione_pof_concordanza} mostra la variazione della concordanza \emph{semplice} tra l'analisi delle controparti di Enlow e l'analisi di Giannì (angolo cranio-occlusale), al variare di $\varepsilon$.

\begin{figure}[ht!]
\centering
% GNUPLOT: LaTeX picture using EEPIC macros
\setlength{\unitlength}{0.120450pt}
\begin{picture}(3000,1800)(0,0)
\footnotesize
\color{black}
\thicklines \path(349,265)(390,265)
\thicklines \path(2896,265)(2855,265)
\put(308,265){\makebox(0,0)[r]{ 30}}
\color{black}
\thicklines \path(349,507)(390,507)
\thicklines \path(2896,507)(2855,507)
\put(308,507){\makebox(0,0)[r]{ 40}}
\color{black}
\thicklines \path(349,749)(390,749)
\thicklines \path(2896,749)(2855,749)
\put(308,749){\makebox(0,0)[r]{ 50}}
\color{black}
\thicklines \path(349,991)(390,991)
\thicklines \path(2896,991)(2855,991)
\put(308,991){\makebox(0,0)[r]{ 60}}
\color{black}
\thicklines \path(349,1233)(390,1233)
\thicklines \path(2896,1233)(2855,1233)
\put(308,1233){\makebox(0,0)[r]{ 70}}
\color{black}
\thicklines \path(349,1475)(390,1475)
\thicklines \path(2896,1475)(2855,1475)
\put(308,1475){\makebox(0,0)[r]{ 80}}
\color{black}
\thicklines \path(349,1717)(390,1717)
\thicklines \path(2896,1717)(2855,1717)
\put(308,1717){\makebox(0,0)[r]{ 90}}
\color{black}
\thicklines \path(349,265)(349,306)
\thicklines \path(349,1717)(349,1676)
\put(349,182){\makebox(0,0){ 0}}
\color{black}
\thicklines \path(713,265)(713,306)
\thicklines \path(713,1717)(713,1676)
\put(713,182){\makebox(0,0){ 2}}
\color{black}
\thicklines \path(1077,265)(1077,306)
\thicklines \path(1077,1717)(1077,1676)
\put(1077,182){\makebox(0,0){ 4}}
\color{black}
\thicklines \path(1441,265)(1441,306)
\thicklines \path(1441,1717)(1441,1676)
\put(1441,182){\makebox(0,0){ 6}}
\color{black}
\thicklines \path(1804,265)(1804,306)
\thicklines \path(1804,1717)(1804,1676)
\put(1804,182){\makebox(0,0){ 8}}
\color{black}
\thicklines \path(2168,265)(2168,306)
\thicklines \path(2168,1717)(2168,1676)
\put(2168,182){\makebox(0,0){ 10}}
\color{black}
\thicklines \path(2532,265)(2532,306)
\thicklines \path(2532,1717)(2532,1676)
\put(2532,182){\makebox(0,0){ 12}}
\color{black}
\thicklines \path(2896,265)(2896,306)
\thicklines \path(2896,1717)(2896,1676)
\put(2896,182){\makebox(0,0){ 14}}
\color{black}
\color{black}
\thicklines \path(349,1717)(349,265)(2896,265)(2896,1717)(349,1717)
\color{black}
\put(102,991){\makebox(0,0){\rotatebox{90}{\%}}}
\color{black}
\color{black}
\put(1622,58){\makebox(0,0){$\varepsilon$}}
\color{black}
\color{black}
\color{red}
\color{black}
\put(2568,1635){\makebox(0,0)[r]{Concordanza}}
\color{red}
\thinlines \path(2609,1635)(2814,1635)
\thinlines \path(349,1125)(349,1125)(531,1208)(713,1291)(895,1125)(1077,1625)(1259,1291)(1441,1125)(1622,791)(1804,373)(1986,373)(2168,373)(2350,373)(2532,373)(2714,373)(2896,290)
\put(349,1125){\makebox(0,0){$\Diamond$}}
\put(531,1208){\makebox(0,0){$\Diamond$}}
\put(713,1291){\makebox(0,0){$\Diamond$}}
\put(895,1125){\makebox(0,0){$\Diamond$}}
\put(1077,1625){\makebox(0,0){$\Diamond$}}
\put(1259,1291){\makebox(0,0){$\Diamond$}}
\put(1441,1125){\makebox(0,0){$\Diamond$}}
\put(1622,791){\makebox(0,0){$\Diamond$}}
\put(1804,373){\makebox(0,0){$\Diamond$}}
\put(1986,373){\makebox(0,0){$\Diamond$}}
\put(2168,373){\makebox(0,0){$\Diamond$}}
\put(2350,373){\makebox(0,0){$\Diamond$}}
\put(2532,373){\makebox(0,0){$\Diamond$}}
\put(2714,373){\makebox(0,0){$\Diamond$}}
\put(2896,290){\makebox(0,0){$\Diamond$}}
\put(2711,1635){\makebox(0,0){$\Diamond$}}
\color{black}
\thicklines \path(349,1717)(349,265)(2896,265)(2896,1717)(349,1717)
\color{black}
\end{picture}

\caption{Concordanza semplice tra Enlow e Giannì sull'inclinazione del piano occlusale}
\label{graph:inclinazione_pof_concordanza}
\end{figure}

Poiché nel campione considerato non è stata rilevata alcuna ante-rotazione occlusale secondo Giannì, il test del $\chi^2$ di Pearson non può essere applicato. Questo è dovuto al fatto che questo test non prevede casi con frequenze nulle (in questo caso, nessuna ante-rotazione).

Il grafico \ref{graph:inclinazione_pof_concordanza} include anche quei casi per cui la concordanza è dovuta solamente al caso. Per eliminare questi casi è stata utilizzata la statistica \emph{$\kappa$ di Fleiss}, così da ottenere una percentuale di concordanza depurata dal caso. Il grafico \vref{graph:inclinazione_pof_kappa} mostra la variazione di questa concordanza \emph{depurata} al variare di $\varepsilon$.

\begin{figure}[ht!]
\centering
% GNUPLOT: LaTeX picture using EEPIC macros
\setlength{\unitlength}{0.120450pt}
\begin{picture}(3000,1800)(0,0)
\footnotesize
\color{black}
\thicklines \path(349,265)(390,265)
\thicklines \path(2896,265)(2855,265)
\put(308,265){\makebox(0,0)[r]{-60}}
\color{black}
\thicklines \path(349,472)(390,472)
\thicklines \path(2896,472)(2855,472)
\put(308,472){\makebox(0,0)[r]{-40}}
\color{black}
\thicklines \path(349,680)(390,680)
\thicklines \path(2896,680)(2855,680)
\put(308,680){\makebox(0,0)[r]{-20}}
\color{black}
\thicklines \path(349,887)(390,887)
\thicklines \path(2896,887)(2855,887)
\put(308,887){\makebox(0,0)[r]{ 0}}
\color{black}
\thicklines \path(349,1095)(390,1095)
\thicklines \path(2896,1095)(2855,1095)
\put(308,1095){\makebox(0,0)[r]{ 20}}
\color{black}
\thicklines \path(349,1302)(390,1302)
\thicklines \path(2896,1302)(2855,1302)
\put(308,1302){\makebox(0,0)[r]{ 40}}
\color{black}
\thicklines \path(349,1510)(390,1510)
\thicklines \path(2896,1510)(2855,1510)
\put(308,1510){\makebox(0,0)[r]{ 60}}
\color{black}
\thicklines \path(349,1717)(390,1717)
\thicklines \path(2896,1717)(2855,1717)
\put(308,1717){\makebox(0,0)[r]{ 80}}
\color{black}
\thicklines \path(349,265)(349,306)
\thicklines \path(349,1717)(349,1676)
\put(349,182){\makebox(0,0){ 0}}
\color{black}
\thicklines \path(713,265)(713,306)
\thicklines \path(713,1717)(713,1676)
\put(713,182){\makebox(0,0){ 2}}
\color{black}
\thicklines \path(1077,265)(1077,306)
\thicklines \path(1077,1717)(1077,1676)
\put(1077,182){\makebox(0,0){ 4}}
\color{black}
\thicklines \path(1441,265)(1441,306)
\thicklines \path(1441,1717)(1441,1676)
\put(1441,182){\makebox(0,0){ 6}}
\color{black}
\thicklines \path(1804,265)(1804,306)
\thicklines \path(1804,1717)(1804,1676)
\put(1804,182){\makebox(0,0){ 8}}
\color{black}
\thicklines \path(2168,265)(2168,306)
\thicklines \path(2168,1717)(2168,1676)
\put(2168,182){\makebox(0,0){ 10}}
\color{black}
\thicklines \path(2532,265)(2532,306)
\thicklines \path(2532,1717)(2532,1676)
\put(2532,182){\makebox(0,0){ 12}}
\color{black}
\thicklines \path(2896,265)(2896,306)
\thicklines \path(2896,1717)(2896,1676)
\put(2896,182){\makebox(0,0){ 14}}
\color{black}
\color{black}
\thicklines \path(349,1717)(349,265)(2896,265)(2896,1717)(349,1717)
\color{black}
\put(102,991){\makebox(0,0){\rotatebox{90}{\%}}}
\color{black}
\color{black}
\put(1622,58){\makebox(0,0){$\varepsilon$}}
\color{black}
\color{black}
\color{red}
\color{black}
\put(2568,1635){\makebox(0,0)[r]{$\kappa$ di Fleiss}}
\color{red}
\thinlines \path(2609,1635)(2814,1635)
\thinlines \path(349,784)(349,784)(531,960)(713,1115)(895,1074)(1077,1624)(1259,1364)(1441,1230)(1622,939)(1804,493)(1986,493)(2168,493)(2350,493)(2532,493)(2714,493)(2896,389)
\put(349,784){\makebox(0,0){$\Diamond$}}
\put(531,960){\makebox(0,0){$\Diamond$}}
\put(713,1115){\makebox(0,0){$\Diamond$}}
\put(895,1074){\makebox(0,0){$\Diamond$}}
\put(1077,1624){\makebox(0,0){$\Diamond$}}
\put(1259,1364){\makebox(0,0){$\Diamond$}}
\put(1441,1230){\makebox(0,0){$\Diamond$}}
\put(1622,939){\makebox(0,0){$\Diamond$}}
\put(1804,493){\makebox(0,0){$\Diamond$}}
\put(1986,493){\makebox(0,0){$\Diamond$}}
\put(2168,493){\makebox(0,0){$\Diamond$}}
\put(2350,493){\makebox(0,0){$\Diamond$}}
\put(2532,493){\makebox(0,0){$\Diamond$}}
\put(2714,493){\makebox(0,0){$\Diamond$}}
\put(2896,389){\makebox(0,0){$\Diamond$}}
\put(2711,1635){\makebox(0,0){$\Diamond$}}
\color{black}
\thicklines \path(349,1717)(349,265)(2896,265)(2896,1717)(349,1717)
\color{black}
\end{picture}

\caption{Concordanza depurata tra Enlow e Giannì sull'inclinazione del piano occlusale}
\label{graph:inclinazione_pof_kappa}
\end{figure}

Com'è possibile desumere da quest'ultimo grafico, gli unici valori di $\varepsilon$ per cui $\kappa$ è positivo (e quindi la concordanza non è tutta dovuta al caso) sono quelli tra 1 e 7 (compresi). Tra questi, un $\varepsilon = 4$ mostra la più alta percentuale di concordanza: 86,21\% quella semplice, e 71,95\% quella depurata dal caso.

Le diagnosi non coincidono in $4$ pazienti: \#3, \#5, \#18 e \#19.

Il paziente \#5 (immagini a pagina \pageref{paz:EMALO2001}) presenta un angolo cranio-occlusale di $18°$, quindi di solo $1°$ al di fuori del range di normalità (secondo Giannì). Includendolo tra i pazienti normo-ruotati, le concordanze salirebbero a 89,66\% e 79,26\%, rispettivamente.

Il paziente \#18 (immagini a pagina \pageref{paz:TILO1999}) ha un angolo cranio-occlusale di $13°$, e un $\varepsilon = 0,\overline{22}$, ma la componente verticale anteriore è $0,5 mm$ più corta rispetto alla componente verticale posteriore. Questo è verosimilmente dovuto all'eccessiva post-inclinazione del piano craniale di riferimento \piano{FMS}{Se}, che falsa l'effettiva inclinazione del piano occlusale. Poiché la differenza è comunque minima, anche questo paziente può essere incluso nel range di normalità (e, nuovamente, le concordanze tra le due analisi salirebbero rispettivamente a 93,10\% e 85,78\%).

Gli altri due pazienti (\#3 e \#19, pagina \pageref{paz:MASCHI2000} e \pageref{paz:ELITRI1998} rispettivamente) presentano diagnosi effettivamente diverse: normo-inclinati per Enlow, post-ruotati per Giannì ($21°$ in entrambi i casi).

È possibile quindi concludere che, considerando un $\varepsilon = 4$ in Enlow (cioè, considerando il \punto{PM} come medio rispetto agli altri due, la differenza tra \punto{CVA} e \punto{CVP} è di 12 mm), esiste un \emph{sostanziale accordo} tra le due analisi nella valutazione dell'inclinazione del piano occlusale.

\section{Dimensioni mascellari}
Questo parametro è stato valutato nelle analisi di Giannì e di Coben. Per il primo, la dimensione sagittale del mascellare superiore è data dalla distanza millimetrica \piano{SNP}{A}, pari a $41 mm$ a 4 anni, con una crescita di $0,5 mm$ annui, fino ad un massimo di $46 mm$ a crescita terminata. Per Coben, tale dimensione è rappresentata della proiezione di \piano{Ptm}{A} sul \punto{BaH}, e viene rapportata in percentuale alla base cranica \piano{Ba}{N} ($51,40 \pm 3,00\%$). In entrambi i casi, valori superiori denotano ipermaxillia, valori inferiori depongono per un'ipomaxillia.

\begin{table}
\centering
\caption{Frequenza delle dimensioni mascellari}
\label{tab:dimensioni_mascellari_frequenze}
\begin{tabular}{>{\bfseries}ccc}
\toprule
& \textbf{Giannì} & \textbf{Coben} \\
\midrule
Normo-maxillia & 1 & 22 \\
Ipomaxillia & 10 & 0 \\
Ipermaxillia & 18 & 7 \\
\bottomrule
\end{tabular}
\end{table}

La tabella delle frequenze \vref{tab:dimensioni_mascellari_frequenze} mostra già notevoli differenze: Giannì trova una sola normo-maxillia, Coben ne trova ben 22 (su un totale di 29 pazienti). Le due analisi concordano solamente in cinque casi (pazienti \#1, \#8, \#9, \#15 e \#28), e in tre casi danno risultati opposti (ipomaxillia in un caso e ipermaxillia nell'altro, pazienti \#16, \#20 e \#23). La concordanza semplice risulta quindi essere solamente del $17,24\%$, e anche la $\kappa$ di Fleiss indica una grave discordanza, essendo pari al $-31,94\%$. Questa discrepanza è probabilmente dovuta al fatto che, diversamente da altri parametri, in questo caso Giannì fornisce un valore millimetrico preciso, senza alcuna deviazione standard, per cui non esiste un vero \emph{range} di normalità, ma piuttosto una misura precisa.

Tentando di ovviare a questo problema, è possibile introdurre sperimentalmente una \emph{deviazione standard} nell'analisi di Giannì, da 0 a 10, con incrementi di $0,5 mm$. La frequenza risultante è espressa nel grafico \vref{graph:dimensioni_mascellari_gianni_ds}: con deviazione standard pari a $10 mm$ vengono diagnosticate tutte normo-maxillie.

\begin{figure}[ht!]
\centering
% GNUPLOT: LaTeX picture using EEPIC macros
\setlength{\unitlength}{0.120450pt}
\begin{picture}(3000,1800)(0,0)
\footnotesize
\color{black}
\thicklines \path(349,265)(390,265)
\thicklines \path(2896,265)(2855,265)
\put(308,265){\makebox(0,0)[r]{ 0}}
\color{black}
\thicklines \path(349,507)(390,507)
\thicklines \path(2896,507)(2855,507)
\put(308,507){\makebox(0,0)[r]{ 5}}
\color{black}
\thicklines \path(349,749)(390,749)
\thicklines \path(2896,749)(2855,749)
\put(308,749){\makebox(0,0)[r]{ 10}}
\color{black}
\thicklines \path(349,991)(390,991)
\thicklines \path(2896,991)(2855,991)
\put(308,991){\makebox(0,0)[r]{ 15}}
\color{black}
\thicklines \path(349,1233)(390,1233)
\thicklines \path(2896,1233)(2855,1233)
\put(308,1233){\makebox(0,0)[r]{ 20}}
\color{black}
\thicklines \path(349,1475)(390,1475)
\thicklines \path(2896,1475)(2855,1475)
\put(308,1475){\makebox(0,0)[r]{ 25}}
\color{black}
\thicklines \path(349,1717)(390,1717)
\thicklines \path(2896,1717)(2855,1717)
\put(308,1717){\makebox(0,0)[r]{ 30}}
\color{black}
\thicklines \path(349,265)(349,306)
\thicklines \path(349,1717)(349,1676)
\put(349,182){\makebox(0,0){ 0}}
\color{black}
\thicklines \path(858,265)(858,306)
\thicklines \path(858,1717)(858,1676)
\put(858,182){\makebox(0,0){ 2}}
\color{black}
\thicklines \path(1368,265)(1368,306)
\thicklines \path(1368,1717)(1368,1676)
\put(1368,182){\makebox(0,0){ 4}}
\color{black}
\thicklines \path(1877,265)(1877,306)
\thicklines \path(1877,1717)(1877,1676)
\put(1877,182){\makebox(0,0){ 6}}
\color{black}
\thicklines \path(2387,265)(2387,306)
\thicklines \path(2387,1717)(2387,1676)
\put(2387,182){\makebox(0,0){ 8}}
\color{black}
\thicklines \path(2896,265)(2896,306)
\thicklines \path(2896,1717)(2896,1676)
\put(2896,182){\makebox(0,0){ 10}}
\color{black}
\color{black}
\thicklines \path(349,1717)(349,265)(2896,265)(2896,1717)(349,1717)
\color{black}
\put(102,991){\makebox(0,0){\rotatebox{90}{pazienti}}}
\color{black}
\color{black}
\put(1622,58){\makebox(0,0){$\sigma$}}
\color{black}
\color{black}
\color{red}
\color{black}
\put(2568,1074){\makebox(0,0)[r]{Normo}}
\color{red}
\thinlines \path(2609,1074)(2814,1074)
\thinlines \path(349,313)(349,313)(476,362)(604,362)(731,507)(858,701)(986,894)(1113,991)(1240,1233)(1368,1330)(1495,1427)(1622,1427)(1750,1475)(1877,1523)(2005,1572)(2132,1572)(2259,1620)(2387,1620)(2514,1620)(2641,1620)(2769,1620)(2896,1669)
\put(349,313){\makebox(0,0){$\Diamond$}}
\put(476,362){\makebox(0,0){$\Diamond$}}
\put(604,362){\makebox(0,0){$\Diamond$}}
\put(731,507){\makebox(0,0){$\Diamond$}}
\put(858,701){\makebox(0,0){$\Diamond$}}
\put(986,894){\makebox(0,0){$\Diamond$}}
\put(1113,991){\makebox(0,0){$\Diamond$}}
\put(1240,1233){\makebox(0,0){$\Diamond$}}
\put(1368,1330){\makebox(0,0){$\Diamond$}}
\put(1495,1427){\makebox(0,0){$\Diamond$}}
\put(1622,1427){\makebox(0,0){$\Diamond$}}
\put(1750,1475){\makebox(0,0){$\Diamond$}}
\put(1877,1523){\makebox(0,0){$\Diamond$}}
\put(2005,1572){\makebox(0,0){$\Diamond$}}
\put(2132,1572){\makebox(0,0){$\Diamond$}}
\put(2259,1620){\makebox(0,0){$\Diamond$}}
\put(2387,1620){\makebox(0,0){$\Diamond$}}
\put(2514,1620){\makebox(0,0){$\Diamond$}}
\put(2641,1620){\makebox(0,0){$\Diamond$}}
\put(2769,1620){\makebox(0,0){$\Diamond$}}
\put(2896,1669){\makebox(0,0){$\Diamond$}}
\put(2711,1074){\makebox(0,0){$\Diamond$}}
\color{blue}
\color{black}
\put(2568,991){\makebox(0,0)[r]{Ipo}}
\color{blue}
\thinlines \path(2609,991)(2814,991)
\thinlines \path(349,749)(349,749)(476,749)(604,749)(731,701)(858,652)(986,555)(1113,507)(1240,507)(1368,459)(1495,362)(1622,362)(1750,362)(1877,362)(2005,362)(2132,362)(2259,313)(2387,313)(2514,313)(2641,313)(2769,313)(2896,265)
\put(349,749){\makebox(0,0){$+$}}
\put(476,749){\makebox(0,0){$+$}}
\put(604,749){\makebox(0,0){$+$}}
\put(731,701){\makebox(0,0){$+$}}
\put(858,652){\makebox(0,0){$+$}}
\put(986,555){\makebox(0,0){$+$}}
\put(1113,507){\makebox(0,0){$+$}}
\put(1240,507){\makebox(0,0){$+$}}
\put(1368,459){\makebox(0,0){$+$}}
\put(1495,362){\makebox(0,0){$+$}}
\put(1622,362){\makebox(0,0){$+$}}
\put(1750,362){\makebox(0,0){$+$}}
\put(1877,362){\makebox(0,0){$+$}}
\put(2005,362){\makebox(0,0){$+$}}
\put(2132,362){\makebox(0,0){$+$}}
\put(2259,313){\makebox(0,0){$+$}}
\put(2387,313){\makebox(0,0){$+$}}
\put(2514,313){\makebox(0,0){$+$}}
\put(2641,313){\makebox(0,0){$+$}}
\put(2769,313){\makebox(0,0){$+$}}
\put(2896,265){\makebox(0,0){$+$}}
\put(2711,991){\makebox(0,0){$+$}}
\color{green}
\color{black}
\put(2568,908){\makebox(0,0)[r]{Iper}}
\color{green}
\thinlines \path(2609,908)(2814,908)
\thinlines \path(349,1136)(349,1136)(476,1088)(604,1088)(731,991)(858,846)(986,749)(1113,701)(1240,459)(1368,410)(1495,410)(1622,410)(1750,362)(1877,313)(2005,265)(2132,265)(2259,265)(2387,265)(2514,265)(2641,265)(2769,265)(2896,265)
\put(349,1136){\makebox(0,0){$\Box$}}
\put(476,1088){\makebox(0,0){$\Box$}}
\put(604,1088){\makebox(0,0){$\Box$}}
\put(731,991){\makebox(0,0){$\Box$}}
\put(858,846){\makebox(0,0){$\Box$}}
\put(986,749){\makebox(0,0){$\Box$}}
\put(1113,701){\makebox(0,0){$\Box$}}
\put(1240,459){\makebox(0,0){$\Box$}}
\put(1368,410){\makebox(0,0){$\Box$}}
\put(1495,410){\makebox(0,0){$\Box$}}
\put(1622,410){\makebox(0,0){$\Box$}}
\put(1750,362){\makebox(0,0){$\Box$}}
\put(1877,313){\makebox(0,0){$\Box$}}
\put(2005,265){\makebox(0,0){$\Box$}}
\put(2132,265){\makebox(0,0){$\Box$}}
\put(2259,265){\makebox(0,0){$\Box$}}
\put(2387,265){\makebox(0,0){$\Box$}}
\put(2514,265){\makebox(0,0){$\Box$}}
\put(2641,265){\makebox(0,0){$\Box$}}
\put(2769,265){\makebox(0,0){$\Box$}}
\put(2896,265){\makebox(0,0){$\Box$}}
\put(2711,908){\makebox(0,0){$\Box$}}
\color{black}
\thicklines \path(349,1717)(349,265)(2896,265)(2896,1717)(349,1717)
\color{black}
\end{picture}

\caption{Variazione delle frequenze delle dimensioni mascellari al variare di $\sigma$}
\label{graph:dimensioni_mascellari_gianni_ds}
\end{figure}

Introducendo una deviazione standard al valore proposto da Giannì, solo la concordanza pura raggiunge valori di più del $70\%$; la $\kappa$ di Fleiss è invece generalmente negativa, o con valori di massimo $8,74\%$ quando positiva. Questo indica che esiste una notevole discrepanza, e che l'accordo tra le due analisi è generalmente dovuto al caso. L'andamento dei due tipi di concordanza è rappresentato nel grafico \vref{graph:dimensioni_mascellari_concordanze}.

\begin{figure}[ht!]
\centering
% GNUPLOT: LaTeX picture using EEPIC macros
\setlength{\unitlength}{0.120450pt}
\begin{picture}(3000,1800)(0,0)
\footnotesize
\color{black}
\thicklines \path(349,265)(390,265)
\thicklines \path(2896,265)(2855,265)
\put(308,265){\makebox(0,0)[r]{-60}}
\color{black}
\thicklines \path(349,472)(390,472)
\thicklines \path(2896,472)(2855,472)
\put(308,472){\makebox(0,0)[r]{-40}}
\color{black}
\thicklines \path(349,680)(390,680)
\thicklines \path(2896,680)(2855,680)
\put(308,680){\makebox(0,0)[r]{-20}}
\color{black}
\thicklines \path(349,887)(390,887)
\thicklines \path(2896,887)(2855,887)
\put(308,887){\makebox(0,0)[r]{ 0}}
\color{black}
\thicklines \path(349,1095)(390,1095)
\thicklines \path(2896,1095)(2855,1095)
\put(308,1095){\makebox(0,0)[r]{ 20}}
\color{black}
\thicklines \path(349,1302)(390,1302)
\thicklines \path(2896,1302)(2855,1302)
\put(308,1302){\makebox(0,0)[r]{ 40}}
\color{black}
\thicklines \path(349,1510)(390,1510)
\thicklines \path(2896,1510)(2855,1510)
\put(308,1510){\makebox(0,0)[r]{ 60}}
\color{black}
\thicklines \path(349,1717)(390,1717)
\thicklines \path(2896,1717)(2855,1717)
\put(308,1717){\makebox(0,0)[r]{ 80}}
\color{black}
\thicklines \path(349,265)(349,306)
\thicklines \path(349,1717)(349,1676)
\put(349,182){\makebox(0,0){ 0}}
\color{black}
\thicklines \path(858,265)(858,306)
\thicklines \path(858,1717)(858,1676)
\put(858,182){\makebox(0,0){ 2}}
\color{black}
\thicklines \path(1368,265)(1368,306)
\thicklines \path(1368,1717)(1368,1676)
\put(1368,182){\makebox(0,0){ 4}}
\color{black}
\thicklines \path(1877,265)(1877,306)
\thicklines \path(1877,1717)(1877,1676)
\put(1877,182){\makebox(0,0){ 6}}
\color{black}
\thicklines \path(2387,265)(2387,306)
\thicklines \path(2387,1717)(2387,1676)
\put(2387,182){\makebox(0,0){ 8}}
\color{black}
\thicklines \path(2896,265)(2896,306)
\thicklines \path(2896,1717)(2896,1676)
\put(2896,182){\makebox(0,0){ 10}}
\color{black}
\color{black}
\thicklines \path(349,1717)(349,265)(2896,265)(2896,1717)(349,1717)
\color{black}
\put(102,991){\makebox(0,0){\rotatebox{90}{\%}}}
\color{black}
\color{black}
\put(1622,58){\makebox(0,0){$\sigma$}}
\color{black}
\color{black}
\color{red}
\color{black}
\put(2568,431){\makebox(0,0)[r]{Concordanza semplice}}
\color{red}
\thinlines \path(2609,431)(2814,431)
\thinlines \path(349,1066)(349,1066)(476,1030)(604,1030)(731,1030)(858,1173)(986,1281)(1113,1352)(1240,1459)(1368,1531)(1495,1603)(1622,1603)(1750,1567)(1877,1603)(2005,1638)(2132,1638)(2259,1638)(2387,1638)(2514,1638)(2641,1638)(2769,1638)(2896,1674)
\put(349,1066){\makebox(0,0){$\Diamond$}}
\put(476,1030){\makebox(0,0){$\Diamond$}}
\put(604,1030){\makebox(0,0){$\Diamond$}}
\put(731,1030){\makebox(0,0){$\Diamond$}}
\put(858,1173){\makebox(0,0){$\Diamond$}}
\put(986,1281){\makebox(0,0){$\Diamond$}}
\put(1113,1352){\makebox(0,0){$\Diamond$}}
\put(1240,1459){\makebox(0,0){$\Diamond$}}
\put(1368,1531){\makebox(0,0){$\Diamond$}}
\put(1495,1603){\makebox(0,0){$\Diamond$}}
\put(1622,1603){\makebox(0,0){$\Diamond$}}
\put(1750,1567){\makebox(0,0){$\Diamond$}}
\put(1877,1603){\makebox(0,0){$\Diamond$}}
\put(2005,1638){\makebox(0,0){$\Diamond$}}
\put(2132,1638){\makebox(0,0){$\Diamond$}}
\put(2259,1638){\makebox(0,0){$\Diamond$}}
\put(2387,1638){\makebox(0,0){$\Diamond$}}
\put(2514,1638){\makebox(0,0){$\Diamond$}}
\put(2641,1638){\makebox(0,0){$\Diamond$}}
\put(2769,1638){\makebox(0,0){$\Diamond$}}
\put(2896,1674){\makebox(0,0){$\Diamond$}}
\put(2711,431){\makebox(0,0){$\Diamond$}}
\color{blue}
\color{black}
\put(2568,348){\makebox(0,0)[r]{$\kappa$ di Fleiss}}
\color{blue}
\thinlines \path(2609,348)(2814,348)
\thinlines \path(349,556)(349,556)(476,500)(604,500)(731,471)(858,647)(986,731)(1113,801)(1240,849)(1368,916)(1495,978)(1622,978)(1750,800)(1877,833)(2005,867)(2132,867)(2259,742)(2387,742)(2514,742)(2641,742)(2769,742)(2896,745)
\put(349,556){\makebox(0,0){$+$}}
\put(476,500){\makebox(0,0){$+$}}
\put(604,500){\makebox(0,0){$+$}}
\put(731,471){\makebox(0,0){$+$}}
\put(858,647){\makebox(0,0){$+$}}
\put(986,731){\makebox(0,0){$+$}}
\put(1113,801){\makebox(0,0){$+$}}
\put(1240,849){\makebox(0,0){$+$}}
\put(1368,916){\makebox(0,0){$+$}}
\put(1495,978){\makebox(0,0){$+$}}
\put(1622,978){\makebox(0,0){$+$}}
\put(1750,800){\makebox(0,0){$+$}}
\put(1877,833){\makebox(0,0){$+$}}
\put(2005,867){\makebox(0,0){$+$}}
\put(2132,867){\makebox(0,0){$+$}}
\put(2259,742){\makebox(0,0){$+$}}
\put(2387,742){\makebox(0,0){$+$}}
\put(2514,742){\makebox(0,0){$+$}}
\put(2641,742){\makebox(0,0){$+$}}
\put(2769,742){\makebox(0,0){$+$}}
\put(2896,745){\makebox(0,0){$+$}}
\put(2711,348){\makebox(0,0){$+$}}
\color{black}
\thicklines \path(349,1717)(349,265)(2896,265)(2896,1717)(349,1717)
\color{black}
\end{picture}

\caption{Variazione delle concordanze (pura e $\kappa$ di Fleiss) al variare di $\sigma$}
\label{graph:dimensioni_mascellari_concordanze}
\end{figure}
