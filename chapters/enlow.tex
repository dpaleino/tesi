\chapter{Analisi delle controparti di Enlow}
Quest'analisi rappresenta il metodo cefalometrico secondo Donald Enlow, in cui le varie parti facciali e craniche vengono paragonate le une alle altre: il soggetto in esame viene confrontato con sé stesso, e non con la media della popolazione.\\

Prima di descrivere il metodo dettagliatamente è necessario soffermarsi su due termini: \emph{dimensione} (orizzontale e verticale) e \emph{allineamento} (tipo di rotazione) dell'osso.

La \emph{dimensione} rappresenta la misura assoluta di una lunghezza; una determinata zona può essere lunga o corta rispetto al suo combaciamento con le altre parti vicine.

L'\emph{allineamento} rappresenta la misura relativa di una lunghezza che, proiettata su un piano di riferimento, ha subito una rotazione; qualsiasi movimento di rotazione infatti può aumentare o diminuire la misura della proiezione di una dimensione.

Pertanto, per poter analizzare tra loro due controparti non è sufficiente conoscerne la dimensione, ma è necessario anche valutarne l'allineamento e comprendere quanto questo influisca sulle loro effettive dimensioni.

Il principio razionale di questa analisi è il confronto tra la dimensione verticale e/o la dimensione orizzontale di una parte con la sua controparte specifica. Se esse corrispondono, esiste un equilibrio dimensionale; al contrario se divergono, lo squilibrio che ne risulta pu causare un effetto di retrusione o di protrusione della parte coinvolta.\\

Nell'analisi delle controparti di Enlow sono previste due fasi:
\begin{enumerate}
\item una \emph{statica}, in cui ogni parte è paragonata alla sua controparte senza considerare le medie della popolazione (tracciato funzionale);
\item una \emph{dinamica}, in cui il soggetto viene confrontato con un tracciato ideale.
\end{enumerate}

\section{Punti di repere}
\section{Linee, piani ed angoli della base cranica}
\section{Linee, piani ed angoli della base cranica nel tracciato neutro}
\section{Analisi statica del tracciato cefalometrico}
\subsection{Analisi dell'equilibrio verticale}
\subsection{Analisi dell'equilibrio orizzontale}
\section{Analisi dinamica del tracciato cefalometrico}
\section{Casi clinici}