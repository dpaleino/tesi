\chapter{Analisi delle controparti di Enlow}
Quest'analisi rappresenta il metodo cefalometrico secondo Donald Enlow, in cui le varie parti facciali e craniche vengono paragonate le une alle altre: il soggetto in esame viene confrontato con sé stesso, e non con la media della popolazione.

Prima di descrivere il metodo dettagliatamente è necessario soffermarsi su due termini: \emph{dimensione} (orizzontale e verticale) e \emph{allineamento} (tipo di rotazione) dell'osso.

La \emph{dimensione} rappresenta la misura assoluta di una lunghezza; una determinata zona può essere lunga o corta rispetto al suo combaciamento con le altre parti vicine.

L'\emph{allineamento} rappresenta la misura relativa di una lunghezza che, proiettata su un piano di riferimento, ha subito una rotazione; qualsiasi movimento di rotazione infatti può aumentare o diminuire la misura della proiezione di una dimensione.

Pertanto, per poter analizzare tra loro due controparti non è sufficiente conoscerne la dimensione, ma è necessario anche valutarne l'allineamento e comprendere quanto questo influisca sulle loro effettive dimensioni.

Il principio razionale di questa analisi è il confronto tra la dimensione verticale e/o la dimensione orizzontale di una parte con la sua controparte specifica. Se esse corrispondono, esiste un equilibrio dimensionale; al contrario se divergono, lo squilibrio che ne risulta può causare un effetto di retrusione o di protrusione della parte coinvolta.\\

Nell'analisi delle controparti di Enlow sono previste due fasi:
\begin{enumerate}
\item una \emph{statica}, in cui ogni parte è paragonata alla sua controparte senza considerare le medie della popolazione (tracciato funzionale);
\item una \emph{dinamica}, in cui il soggetto viene confrontato con un tracciato ideale.
\end{enumerate}

\section{Punti di repere}
Oltre ai \textit{classici} punti di repere cefalometrici (Sottospinale, Sopramentale, Menton, \ldots{}), Enlow utilizzò dei punti cefalometrici propri.
\paragraph{Fronto-Mascellare (\punto{FM})} punto mediano di unione tra osso frontale, osso mascellare e osso nasale. Corrisponde al margine posteriore della sutura fronto-nasale (il cui margine anteriore è rappresentato da \punto{N}).
\paragraph{Sutura sfeno-etmoidale (\punto{Se})} punto mediano di intersezione del profilo anteriore della cresta sfenoidale con il pavimento della fossa cranica media.
\paragraph{Fessura pterigo-mascellare (\punto{PTM})} punto più basso della fessura pterigo-mascellare (un'area bilaterale radiotrasparente a forma di goccia).
\paragraph{Articolare di Björk (\punto{Ar})} punto bilaterale di intersezione tra il bordo inferiore del massiccio sfeno-occipitale e la superficie posteriore dei condili. Non rappresenta una struttura ossea, ma è un'immagine radiografica: indica la posizione del condilo nel punto in cui emerge dalla cavità glenoidea.
\paragraph{Contatto molare (\punto{Cm})} e \textbf{contatto molare deciduo (\punto{cm})}, punto bilaterale di contatto molare disto-occlusale delle cuspidi dei primi molari permanenti e decidui.
\paragraph{Prosthion superiore (\punto{SPr})} e \textbf{prosthion inferiore (\punto{IPr})}, punto mediano più sporgente del processo alveolare della mascella e della mandibola, tra gli incisivi centrali.
\paragraph{Tuberosità linguale (\punto{LT})} punto bilaterale ottenuto dall'intersezione del piano occlusale con il margine anteriore del ramo mandibolare, rappresenta l'abitacolo per l'ultimo molare (primo, secondo o terzo, secondo la fase di maturazione dentaria).

\section{Punti, piani ed angoli nel tracciato funzionale}
\paragraph{Piano occlusale funzionale (\punto{POF})} linea che unisce l'intercuspidazione dei primi molari nel punto di contatto più occlusale (\punto{Cm} o \punto{cm}) e per il contatto tra i premolari (o molari decidui).
\paragraph{Linea di riferimento (\punto{REF})} retta parallela al \punto{POF}, passante per \punto{Ar}. Rappresenta l'asse di riferimento su cui vengono proiettate alcune delle componenti orizzontali per poter essere confrontate tra loro.
\paragraph{Piano orizzontale della base cranica (\piano{Se}{FM})} piano passante per i punti \punto{Se} e \punto{FM}. Viene utilizzato per delineare il margine superiore del complesso naso-mascellare.
\paragraph{Piano verticale pterigo-mascellare (\punto{PM})} piano verticale passante per \punto{Se} e per \punto{PTM}. Rappresenta un confine strategico tra il complesso naso-mascellare, la base cranica e il faringe. Delimita il margine posteriore del mascellare superiore e viene usato per misurare l'altezza posteriore del complesso naso-mascellare.
\paragraph{Piano verticale naso-mascellare anteriore (\punto{CVA})} parallelo a \punto{PM}, passante per \punto{FM}. Delimita il margine anteriore del complesso naso-mascellare, e viene usato per misurarne l'altezza anteriore.
\paragraph{Piano verticale pavimento cranico-ramo (\punto{CVP})} parallelo a \punto{PM}, passante per \punto{Ar}. Delimita il margine posteriore della base cranica e del ramo mandibolare, e viene usato per misurare l'altezza posteriore di tale complesso.
\paragraph{Pavimento cranico posteriore (\punto{PCF})} unisce \punto{Se} e \punto{Ar}, rappresenta il pavimento della base cranica media e posteriore.
\paragraph{Allineamento posteriore del ramo (\punto{PRA})} distanza tra \punto{Ar} e \punto{POF} nella sua intersezione con il margine posteriore del ramo mandibolare. Rappresenta il limite posteriore del ramo.
\paragraph{Allineamento anteriore del ramo (\punto{ARA})} parallelo a \punto{PRA}, ha come estremità l'asse \punto{REF} e \punto{POF} nel suo punto d'intersezione con il margine anteriore del ramo (punto \punto{LT}).
\paragraph{Mascellare basale (\piano{A}{PM})} distanza tra \punto{A} e \punto{PM}, tracciata parallelamente all'asse \punto{REF}. Rappresenta la lunghezza del mascellare a livello basale.
\paragraph{Mascellare dento-alveolare (\piano{SPr}{PM})} distanza tra \punto{SPr} e \punto{PM}, tracciata parallelamente all'asse \punto{REF}. Rappresenta la lunghezza del mascellare a livello dento-alveolare.
\paragraph{Corpo mandibolare basale (\punto{B}$\perp$\piano{REF}{ARA})} distanza tra la proiezione ortogonale di \punto{B} su \punto{REF} (\punto{B}$\perp$\punto{REF}) e \punto{ARA} nel suo punto di intersezione con \punto{REF}. Rappresenta la lunghezza del corpo mandibolare a livello basale.
\paragraph{Corpo mandibolare dento-alveolare (\punto{IPr}$\perp$\piano{REF}{ARA})} come il precedente, utilizzando \punto{IPr} invece di \punto{B}. Rappresenta la lunghezza del corpo mandibolare a livello dento-alveolare.
\paragraph{Angolo della base cranica} formato dall'intersezione tra il pavimento della base cranica posteriore \punto{PCF} e il piano verticale \punto{PM}, letto nel punto \punto{Se} (angolo \punto{Ar}$\widehat{Se}$\punto{PTM}).
\section{Punti, piani ed angoli nel tracciato neutro}
\paragraph{Sfeno-etmoidale neutro (\punto{Sen})} punto di una circonferenza con centro in \punto{Ar} e raggio uguale a \punto{PCF}, in cui si ottiene un angolo della base cranica uguale a $40.3°$.
\paragraph{Piano verticale pterigo-mascellare neutro (\punto{PMn})} retta parallela a \punto{PM} tale da formare in \punto{Sen} un angolo ideale di $40.3°$ con il pavimento cranico posteriore neutro (\punto{PCFn}).
\paragraph{Pavimento cranico posteriore neutro (\punto{PCFn})} raggio del cerchio che ha centro in \punto{Ar} e che forma con \punto{PMn} un angolo di $40.3°$ nel punto \punto{Sen}.
\paragraph{Angolo della base cranica neutro} in condizioni ideali, deve avere un valore di $40.3°$.
\paragraph{Piano occlusale neutro (\punto{POn})} retta perpendicolare a \punto{PMn}, passante per l'in\-ter\-cu\-spi\-da\-zio\-ne dei primi molari nel punto di contatto interocclusale più distale (\punto{Cm} o \punto{cm}).
\paragraph{Gonion neutro (\punto{Gon})} punto di mezzo, individuato lungo la tangente al bordo inferiore del corpo mandibolare, tra il piano \punto{PMn} e il piano verticale \punto{CVP}.
\section{Analisi statica del tracciato cefalometrico}
La lettura del tracciato statico prevede due fasi:
\begin{itemize}
\item analisi dell'equilibrio verticale
\item analisi dell'equilibrio verticale
\end{itemize}

In una fase successiva si analizzerà poi il tracciato neutro (\textit{analisi dinamica}).

\subsection*{Analisi dell'equilibrio verticale}
L'equilibrio verticale viene valutato mettendo a confronto tra loro le controparti verticali. Tali controparti sono costituite da:
\begin{itemize}
\item \punto{CVA}, componente verticale anteriore;
\item \punto{PM}, componente verticale media;
\item \punto{CVP}, componente verticale posteriore.
\end{itemize}
\paragraph{Componente verticale anteriore (\punto{CVA})} considerata come la distanza tra \punto{FM} e \punto{POF}, misurata lungo il piano verticale naso-mascellare anteriore \punto{CVA}. Rappresenta la dimensione verticale della porzione anteriore del complesso naso-mascellare.
\paragraph{Componente verticale media (\punto{PM})} considerata come la distanza tra \punto{Se} e \punto{POF}, misurata lungo il piano verticale pterigo-mascellare \punto{PM}. Rappresenta la dimensione verticale della porzione posteriore del complesso naso-mascellare.
\paragraph{Componente verticale posteriore (\punto{CVP})} considerata come la distanza tra il piano \piano{Se}{FM} e \punto{POF}, misurata lungo il piano verticale del pavimento cranico-ramo \punto{CVP}. Rappresenta la dimensione verticale del complesso cranio-ramo.

\paragraph{Valutazione dell'equilibrio}
Le tre misurazioni poste a confronto sono ritenute in equilibrio fisiologico quando la differenza tra loro è minima, e comunque a favore della componente verticale anteriore \punto{CVA}; si definisce inoltre armonica la condizione in cui il piano orizzontale \piano{Se}{FM}, prolungato posteriormente, sfiore i processi clinoidei.

Uno squilibrio verticale si realizza quando anche una sola delle tre componenti risulta troppo corta, o troppo lunga, rispetto ad una condizione ideale:

\begin{enumerate}
\item una \punto{CVA} ridotta coincide con un piano bispinale in antero-rotazione;
\item una \punto{CVA} ridotta, in concomitanza con una \punto{PM} ridotta, causa un'antero-rotazione mandibolare, con chiusura dell'angolo goniaco;
\item una \punto{PM} lunga, in concomitanza o meno con una \punto{CVA} lunga, causa una post-rotazione mandibolare, con apertura dell'angolo goniaco.
\end{enumerate}

\subsection*{Analisi dell'equilibrio orizzontale}
L'equilibrio orizzontale è valutato mettendo a confronto le controparti orizzontali tra loro:
\begin{itemize}
\item \textit{controparti orizzontali posteriori}
\begin{itemize}
\item base cranica (\punto{PCF})
\item ramo mandibolare (\piano{PRA}{ARA})
\end{itemize}
\item \textit{controparti orizzontali anteriori}
\begin{itemize}
\item mascellare superiore basale (\piano{A}{PM})
\item mascellare superiore dento-alveolare (\piano{SPr}{PM})
\item corpo mandibolare basale (\punto{B}$\perp$\piano{REF}{ARA})
\item corpo mandibolare dento-alveolare (\punto{IPr}$\perp$\piano{REF}{ARA})
\end{itemize}
\end{itemize}

\paragraph{Base cranica} rappresentata dalla proiezione ortogonale di \punto{PCF} sull'asse di riferimento \punto{REF}.
\paragraph{Ramo mandibolare} rappresentata dalla distanza lungo l'asse \punto{REF} tra i segmenti \punto{PRA} e \punto{ARA}.
\paragraph{Mascellare basale} rappresentata dalla distanza tra il punto \punto{A} e la verticale pterigo-mascellare \punto{PM}, tracciata parallelamente all'asse \punto{REF}.
\paragraph{Mascellare dento-alveolare} come la misura precedente, utilizzando il punto \punto{SPr} al posto di \punto{A}.
\paragraph{Corpo mandibolare basale} rappresentata dalla distanza tra la proiezione ortogonale del punto \punto{B} sull'asse \punto{REF} e \punto{ARA} nel suo punto d'intersezione con \punto{REF}.
\paragraph{Corpo mandibolare dento-alveolare} come la misura precedente, utilizzando il punto \punto{IPr} al posto di \punto{B}.

\paragraph{Valutazione dell'equilibrio}
È necessario mettere a confronto separatamente le misurazioni, espresse in millimetri, delle controparti orizzontali posteriori e di quelle anteriori. Bisogna quindi valutare:
\begin{itemize}
\item la base cranica e il ramo mandibolare;
\item il mascellare e il corpo mandibolare basali;
\item il mascellare e il corpo mandibolare dento-alveolari.
\end{itemize}
Tale confronto viene effettuato utilizzando il metodo della ``differenza millimetrica'', che considera armonica e ideale una condizione in cui la differenza tra parte e controparte varia da 0 a 2mm. In alternativa, Tollaro\nocite{Tollaro1981} ha utilizzato il ``metodo del coefficiente'', che valuta il rapporto tra parte e controparte, e in cui si ha una condizione di equilibrio quando il coefficiente è uguale a uno.

Quando la base cranica è in equilibrio con il ramo mandibolare, e il mascellare superiore con la mandibola, si ottiene un equilibrio \textit{sagittale} riconducibile ad una Classe I scheletrica.

Si realizza uno squilibrio orizzontale quando anche una sola delle controparti risulta troppo corta, o troppo lunga, rispetto alla condizione ideale:

\begin{enumerate}
\item se il corpo mandibolare è piccolo, o il ramo mandibolare è stretto, si realizza una malocclusione di Classe II a componente mandibolare;
\item se il mascellare superiore è grande, o la base cranica è larga, si realizza una malocclusione di Classe II a componente mascellare;
\item se il corpo mandibolare è grande, o il ramo mandibolare è largo, si realizza una malocclusione di Classe III a componente mandibolare;
\item se il mascellare superiore è piccolo, o la base cranica è stretta, si realizza una malocclusione di Classe III a componente mascellare.
\end{enumerate}

Esistono casi in cui si realizza un equilibrio sagittale anche in presenza di squilibri tra le singole parti e controparti, attraverso meccanismi di compenso.

\begin{enumerate}
\item se il corpo mandibolare è piccolo, un ramo largo compenserà lo squilibrio (e viceversa);
\item se il mascellare superiore è piccolo, una base cranica larga compenserà lo squilibrio (e viceversa).
\end{enumerate}

\section{Analisi dinamica del tracciato cefalometrico}
L'analisi delle controparti termina con il confronto tra il tracciato statico precedentemente descritto, definito come \textit{tracciato funzionale}, e un tracciato cosiddetto \textit{neutro}. In questa fase viene inserito il \textit{fattore rotazionale verticale} di tre strutture: base cranica, piano occlusale e ramo mandibolare.

Nell'analisi del tracciato si considerano:

\begin{itemize}
\item l'angolo della base cranica, misurato in \angolo{Se};
\item il piano occlusale funzionale \punto{POF};
\item il punto \punto{Go}.
\end{itemize}

La valutazione di eventuali rotazioni si esegue attraverso la costruzione delle corrispondenti posizioni ``neutre'', e il confronto tra queste e quelle proprie del paziente. Si considerano quindi ideali:

\begin{enumerate}
\item un angolo della base cranica di 40.3°;
\item un \punto{Gon} localizzato a metà tra il piano verticale posteriore \punto{CVP} e il piano verticale \punto{PM};
\item un \punto{POn} perpendicolare al piano verticale neutro \punto{PMn}.
\end{enumerate}

L'unico valore di riferimento è quindi l'angolo ideale della base cranica, di 40.3°. Si esegue quindi la costruzione del tracciato neutro, e si sovrappone su quello funzionale, permettendo così di individuare graficamente il movimento di rotazione delle tre strutture prese in considerazione. Se i due tracciati risultano essere sovrapposti, si è in presenza di una condizione di equilibrio.

Le condizioni di squilibrio sono:

\begin{enumerate}
\item un angolo della base cranica inferiore a 40.3°, che indica uno scarso sviluppo verticale naso-mascellare, e un effetto di rotazione antioraria dell'angolo goniaco, con protrusione mandibolare di compenso;
\item un angolo della base cranica superiore a 40.3°, che indica un'eccessiva discesa del complesso naso-mascellare, con protrusione del mascellare e un effetto di rotazione oraria dell'angolo goniaco, con retrusione mandibolare;
\item un piano occlusale non perpendicolare a \punto{PM}, che indica una rotazione mandibolare, seguita da una modificazione (apertura/chiusura) dell'angolo goniaco.
\end{enumerate}
