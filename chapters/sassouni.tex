\chapter{Analisi di Sassouni}
L'analisi di Sassouni\footcite{Sassouni1955,Sassouni1969} è una \textit{analisi archiale}, una delle prime a considerare il complesso dentofacciale come un'unica unità all'interno del pattern individuale.

L'analisi di Sassouni, detta da alcuni autori la \textit{rosa dei venti}, si basa sul disegno di quattro archi, con centro \punto{O} (definito come il punto di convergenza della parallalela alla linea sopraorbitale, del piano palatale, del piano occlusale e del piano mandibolare):

\begin{description}
\item[arco anteriore] passante per \punto{N};
\item[arco basale] passante per \punto{A};
\item[arco mediofacciale] passante per \punto{Te};
\item[arco posteriore] passante per \punto{Sp}.
\end{description}

Quest'analisi prevede una \textit{valutazione visiva} delle tendenze indicative di una faccia ben proporzionata:

\begin{enumerate}
\item convergenza delle linee parallele al margine sopraorbitale;
\item convergenza dei piani palatale, occlusale e mandibolare verso il punto \punto{O};
\item l'arco anteriore dovrebbe passare per \punto{SNA}, il margine incisale dell'incisivo superiore e \punto{Pog};
\item l'arco basale dovrebbe passare per il punto \punto{B};
\item l'arco mediofacciale, indicativo della posizione dei primi molari, dovrebbe passare tangente alla superficie mesiale del primo molare superiore, quando l'arco anteriore passa per \punto{SNA};
\item l'arco posteriore, indicativo della posizione posteriore della mandibola, dovrebbe passare da \punto{Go}.
\end{enumerate}

Verticalmente, l'altezza facciale superiore e quella inferiore dovrebbero coincidere, sia anteriormente che posteriormente. La lunghezza \piano{Go}{Pog} dovrebbe essere pari alla base cranica (\piano{Sp}{N}, lungo il raggio), estendendosi tra l'arco anteriore e quello posteriore. L'angolo tra il piano della base cranica e il piano palatale dovrebbe essere uguale all'angolo palatomandibolare.

Quest'analisi fornisce un metodo analitico rapido, semplice e conveniente, sebbene localizzare il punto \punto{O} possa essere talvolta difficile.

\begin{table}[ht]
\centering
\footnotesize
\caption{Punti e piani specifici dell'analisi di Sassouni}
\begin{tabularx}{\columnwidth}{>{\textit}clX}
\toprule
\multicolumn{3}{l}{\textbf{Punti di repere}} \\
\midrule
\punto{O} & Punto O & L'intersezione o il punto di convergenza della parallela al piano sopraorbitale, il piano palatale, il piano occlusale e il piano mandibolare.\\
\punto{Si} & Pavimento della sella & Il punto più basso del contorno della sella turcica. \\
\punto{Sp} & Dorso della sella & Il punto più posteriore sul profilo della sella turcica. \\
\punto{Te} & Temporale & L'intersezione della lamina cribrosa dell'etmoide e la parete anteriore della fossa infratemporale con il punto superiore del tetto dell'orbita. \\
\midrule
\multicolumn{3}{l}{\textbf{Piani}} \\
\midrule
\piano{Me}{Go} & Mention-Gonion & Rappresenta l'estensione della base mandibolare. \\
\piano{Ar}{Go} & Piano del ramo & Rappresenta la lunghezza del ramo mandibolare. \\
\piano{Te}{ClinAnt} & Piano sopraorbitale & Linea tangente a Te e al processo clinoideo anteriore dello sfenoide. \\
 & Parallela al piano sopraorbitale & Linea tangente a Si parallela al piano sopraorbitale. \\
\piano{Or}{ClinAnt} & Piano infraorbitale & Linea tangente ad Or e al processo clinoideo anteriore dello sfenoide. \\
 & Piano ottico & Costruito disegnando la bisettrice dell'angolo formato dai piani sopra e infraorbitale. \\
 & Lunghezza mascellare & Distanza lineare (\textit{mm}) da SNA a SNP. \\
 & Lunghezza mandibolare & Distanza lineare (\textit{mm}) da Go a Pog. \\
\bottomrule
\end{tabularx}
\end{table}
