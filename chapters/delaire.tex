\chapter{Analisi architetturale di Delaire}
L'analisi cefalometrica proposta da Delaire è un'analisi architettonica e strutturale che considera linee verticali e orizontali tracciate a partire da punti di repere anatomici, non convenzionali.\\
Per poter utilizzare tale analisi sarà necessario avere una buona teleradiografia, perché il tracciato prende in esame le strutture cranio-maxillo-rachidee.\\

Si distinguono le strutture craniche da quelle mascellari, le prime individuate dalle sigle da C1 a C4, le seconde da CF1 a CF8, rispettivamente \emph{linee craniche} e \emph{linee cranio-facciali}.\\

\emph{C1} suddivide la parte cranica da quella maxillo-rachidea, e ne indica la sua lunghezza. Viene tracciata fino ad un punto OI (occipitale inferiore), dal quale si traccia la tangente alla parte prominente dell'occipite.\\
C1 si traccia a partire dal punto M, che è il punto di unione delle ossa naso-fronto-mascellare sulla parte inferiore della sutura stessa e passante per il punto di incontro tra il condilo zigomatico e quello temporale, punto CT.\\

\emph{C2}, che indica la lunghezza della volta cranica nella proporzione 80-85\% della lunghezza di C1, si traccia perpendicolarmente a partire dal punto di mezzo M-OI, cioè di C1; nella norma dovrebbe cadere dietro il condilo, punto CP, cioè separare la parte rachidea dalla parte maxillo-mandibolare.\\
La parte mandibolare è a sua volta il 40\% e la mascellare il 60\% del tratto MCP, sempre considerando la norma. Se C2 non è tangente al bordo posteriore del condilo si può fare subito una considerazione sul mancato equilibrio della parte rachidea e maxillo-mandibolare a favore della parte rachidea, per esempio nelle terze classi a crescita verticale o di quella mascellare nei casi di terza classe a crescita orizzontale, nelle retrognazie e nei deep-bite. Così pure la proporzione 40-60\% tra i mascellari può variare ed esprimere squilibri tra le due parti, per esempio una proporzione maggiore a favore del mascellare nelle prognazie superiori.\\

\emph{C3}, tracciata dal punto M, passante per i processi clinoidei posteriori fino all'occipite, è la linea della base cranica, piano di riferimento delle strutture sottostanti.\\

\emph{C4}, linea del clivus, va dalla metà dei processi clinoidei posteriori al bordo postero-superiore del dente dell'epistrogeo. C4 definisce insieme a C3 l'angolo posteriore della base cranica. Il valore si aggira intorno a 114° nelle femmine e 112° nei maschi per l'angolo posteriore, mentre l'angolo anteriore definito da C3 a C1 è di circa 22°.\\

Di tutto il tracciato, la struttura più importante da individuare, tracciare e considerare è il \emph{pilastro anteriore}, che corrisponde ad una struttura anatomica ben definita dall'addensamento trabceolare di rinforzo in corrispondenza della branca anteriore del mascellare. Si tratta di una struttura paramediana ma unica nelle teleradiografie latero-laterali.

Il pilastro anteriore è espresso dalla linea \emph{CF1} tracciata da Fm, cioè un millimetro e mezzo dietro M in corrispondenza della perpendicolare a C3, ababssata dalla cresta di rinforzo sul fondo del seno frontale (talvolta sono individuabili due creste e la perpendicolare a C3 si abbassa a partire dal punto di mezzo tra queste).

Lungo il suo percorso, la linea CF1 incontra in punto orbitale inferiore e il canale palatino anteriore, corrispondente quasi sempre all'apice del canino permanente o all'hipomoclion del canino in formazione.

Questi tre punti, soprattutto nelle terze classi, non sono ben allineati e il pilastro anteriore si fa passare per il punto Fm e orbitale inferiore. Il pilastro anteriore è quello che subisce maggiori alterazioni nelle varie patologie seguendo le alterazioni del mascellare superiore, che può avere un aggetto verso l'esterno normale, aumentato o diminuito. Esistono dunque due pilastri anteriori: quello \emph{reale}, dato dalle strutture che si considerano sulla teleradiografia, e quello \emph{ideale}, che considera lo sviluppo dche il mascellare dovrebbe avere se tutte le strutture fossero nella norma. Dal confronto si ha la conoscenza della deviazione nella patologia considerata.

Calcolare l'aggetto del pilastro anteriore nella norma è facile sapendo che la sua crescita in avanti aumenta con l'età e il suo valore angolare, calcolato tra C1 e CF1, va da 80° nei bambini più piccoli a 90° negli adulti, con valori leggermente inferiori nelle femmine. L'angolo normale ideale sarà tracciato considerando l'età del paziente, o anche aggiungendo al valore dell'angolo della base anteriore del cranio il valore di 65. Inoltre il pilastro ideale è tangente all'apice dell'incisivo inferiore, e taglia la sinfisi tra il terzo medio e quello posteriore.\\

Il primo dato diagnostico ci è riferito dal confronto tra pilastro reale e ideale; il secondo, più importante, è dato dal punto di tangenza del pilastro ideale e la sinfisi. Se cade sul bordo posteriore della sinfisi si può dire che la mandibola è grande; se cade davanti, che è piccola. Però questi valori non solo devono essere correlati tra loro, ma devono essere valutati altri dati, per esempio, la rotazione della mandibola -- perché in caso di rotazione posteriore, una mandibola più grande della norma può esere tagliata da CF1 tra terzo medio e terzo posteriore della sinfisi o nella sua parte anteriore.

Altra situazione da considerare, sempre in relazione a CF1, è una mandibola piccola ma anteposta, che si troverà in posizione di normalità rispetto a CF1. Questo dato sarà controllato dalla posizione del bordo posteriore della mandibola rispetto al bordo anteriore del tratto cervicale.\\

Oltre al pilastro anteriore vanno considerati gli altri due pilastri, il medio e il posteriore.\\

Il medio, o \emph{CF2}, è il segmento che congiunge il bregma al punto Pts. Pts è il punto di mezzo del tratto postero-superiore della fessura pterigo-mascellare che si ottiene tracciando la bisettrice dell'angolo formato da C1 e la retta tangente al processo pterigoideo dello sfenoide.\\
La parte terminale di CF2 divide la mandibola in due parti uguali. Se le due parti non sono uguali ci si può trovare in presenza di una mandibola piccola o retroposta (parte anteriore diminuita o postruotata). Se la parte posteriore è più piccola, la mandibola può essere in protrusione oppure aumentata di volume.\\

Il pilastro posteriore, o \emph{CF3}, si traccia parallelamente a CF2 perché così è orientato nelle strutture normali, partendo dal punto di tangenza di C1 con il condilo, cioè il punto Cp.

Anche del pilastro posteriore se ne tracciano due, uno ideale, parallelo, e uno reale, tangente al bordo posteriore della mandibola, sempre partendo dal punto Cp. Nei casi di normalità di posizione mandibolare i due coincidono, nel caso contrario possono indicare: una mandibola retroposta o anteposta, eccessi di sviluppo della zona dell'angolo mandibolare o un'eccessiva inclinazione della branca montante con apertura dell'angolo.\\

\emph{CF4} è la linea del piano bispinale. Nella norma sono allineati con l'articolazione occipito-rachidea, il dente dell'epistrofeo e l'arco dell'atlante, la spina nasale posteriore e quella anteriore.

In un'età vicina alla pubertà il dente dell'epistrofeo e l'arco dell'atlante sono tagliati nella loro parte superiore, ma in un bambino di giovane età si trovano al di sotto della congiungente i condili occipitali e la linea bispinale. La ragione risiede nel fatto che nel processo di crescita le vertebre cervicali si sviluppano più verso l'alto che il basso. Dunque tutti questi punti possono coincidere o meno. Quando nessun punto coincide con l'articolazione occipito-rachidea si devono tracciare diverse linee passanti per i diversi punti, cioè si costituisce una \emph{banda} e la linea di mezzo sarà quella alla quale ci si riferisce.

Da CF4 risulta evidente la posizione della spina nasale anteriore, che può trovarsi verso l'alto, come nei morsi aperti, o verso il basso, nei morsi chiusi. La stessa sorte può subire la spina nasale posteriore che, a seconda dei movimenti della lingua, può orientarsi verso l'alto o verso il basso.\\

L'altezza facciale è espressa da \emph{CF5}, che è tracciata a partire dal punto SNA, reale se si colloca sul piano bispinale, oppure ideale se si trova sul CF4 ricavato dalla fascia. Da questo punto s'innalza una perpendicolare sino al punto N, o meglio Na', perché la retta giace al di fuori del cranio e, nei casi di ipomaxillia grave, talvolta all'interno.

CF5, misurato da Na' a Met', viene calcolato raddoppiando la distanza SNA-Na' e aggiungendo un nono della misura trovata. Il punto trovato Met' viene riportato sul pilastro ideale, tracciando una perpendicolare da Met' al pilastro stesso, e viene indicato con Met.\\

\emph{CF6} indica il piano mandibolare tracciato da Met e tangente alla squama dell'occipitale. Incontra CF4 in un punto OM. Questa linea indica lo sviluppo del bordo inferiore della mandibola, il grado dello sviluppo del ramo perlopiù diminuito nelle iperdivergenze, lungo per esempio in certe grandi sindromi come l'acromegalia e in certi casi di «long face» a ramo lungo. Evidenzierà anche il grado di post-rotazione mandibolare, il cui bordo eccederà dal piano mandibolare stesso.\\

Il piano occlusale espresso da \emph{CF7} si presenta in due modi: reale e ideale; quello reale corrisponde al piano occlusale di Ricketts, mentre quello ideale corrisponde alla metà della distanza tra SNA e Met', che si congiunge indietro nel punto OM.\\

Rimane da considerare il \emph{CF8}, tracciato parallelo a C3 a partire dalla SNA, reale o ideale, sino a raggiungere la zona dell'angolo che potrà coincidere o no col vertice dell'angolo. Ci dà utili indicazioni sull'angolo e sulla lunghezza del ramo della mandibola e la lunghezza di questo in relazione a quela della branca montante del mascellare. Nella norma, il ramo mandibolare è più lungo del mascellare.\\

La ricchezza degli elementi considerati, e soprattutto il fatto che gli elementi scheletrici siano cranio-maxillo-rachidei, permette una grande elaborazione di dati: correlando gli uni agli altri si può, per via induttiva e deduttiva, porre le basi per una buona diagnosi.

\section{Casi clinici}