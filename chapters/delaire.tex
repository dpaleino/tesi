\chapter{Analisi architetturale di Delaire}
\nocite{Delaire1981,Cudia1991}

L'analisi cefalometrica proposta da Delaire è un'analisi architettonica e strutturale che considera linee verticali e orizontali tracciate a partire da punti di repere anatomici non convenzionali. Essa mira a chiarire il mutuo equilibrio tra le varie strutture ossee del cranio e della faccia.\\
Per poter utilizzare tale analisi sarà necessario avere una buona teleradiografia comprendente l'intera calotta cranica.\\

Si distinguono le strutture craniche da quelle mascellari, le prime individuate dalle sigle da C1 a C4, le seconde da CF1 a CF8, rispettivamente \emph{linee craniche} e \emph{linee cranio-facciali}.\\

\section{Analisi del cranio}
\paragraph{Linea di base craniofacciale} o \textbf{C1}, rappresenta il limite craniofacciale, e viene tracciata dal punto \punto{FMS} (sutura fronto-mascellare), passante da \punto{CT} (punto temporo-condilare), ossia il punto inferiore della superficie postero-inferiore del tubercolo articolare del temporale. Tale linea viene poi estesa fino al punto \punto{OI} (occipitale inferiore), che è l'intersezione tra \textbf{C1} e la sua perpendicolare tangente alla superficie esterna dell'osso occipitale.\\
L'intersezione tra \textbf{C1} e la superficie posteriore del condilo definisce il punto \punto{CP}. Normalmente, \punto{CP} è posizionato a metà strada tra \punto{FMS} e \punto{OI}. Questo divide \textbf{C1} in due segmenti uguali, denominati rispettivamente \emph{area craniospinale} e \emph{area craniofacciale}. Il tratto \piano{FMS}{CP} interseca la parte superiore della fessura pterigomascellare in \punto{Pts}. In pazienti con un buon equilibrio cefalometrico, \piano{FMS}{Pts} e \piano{Pts}{CP} stanno in un rapporto di 3:2.

\paragraph{Altezza cranica} o \textbf{C2}, è perpendicolare a \textbf{C1} nel suo punto centrale (normalmente \punto{CP}), e interseca la calotta cranica in un punto \punto{SC}. Normalmente, \punto{SC} è il punto cranico più distante da \textbf{C1}, e \punto{SC} risulta essere la sommità del cranio. La lunghezza di \textbf{C2} è normalmente tra il 75 e l'85\% di \textbf{C1}.\\
Se \textbf{C2} non è tangente al margine posteriore del condilo si può fare subito una considerazione sul mancato equilibrio tra l'area craniospinale e l'area craniofacciale: se passa al davanti sarà a favore della prima, per esempio nelle terze classi a crescita verticale; se passa al di dietro sarà a favore della seconda, ad esempio nei casi di terza classe a crescita orizzontale, nelle retrognazie e nei deep-bite.

\paragraph{Linea superiore della base cranica} o \textbf{C3}, viene tracciata dal punto \punto{FMS}, passante per l'apice del processo clinoideo \punto{Clp}, viene estesa posteriormente fino alla superficie esterna dell'osso occipitale, al punto \punto{OP} (punto occipitale posteriore). Di norma il segmento \piano{FMS}{Clp} è approssimativamente parallelo alla lamina cribrosa dell'etmoide, e passa vicino al processo clinoideo anteriore e il tubercolo ipofisario; il punto \punto{OP} risulta essere molto vicino alla perpendicolare a \textbf{C1} registrata in \punto{OI}.\\
L'angolo tra questa linea e \textbf{C1} è normalmente di 22°.

\paragraph{Linea del clivus} o \textbf{C4}, è formata dal punto \punto{Clp} alla parte postero-inferiore o all'apice del dente dell'epistrofeo. Normalmente, questa linea risulta essere tangente alla sella turcica, la superficie cerebrale dell'osso baso-occipitale e il basion, ed è molto vicina, e a volte tangente, alla superficie postero-superiore del condilo mandibolare.

\section{Analisi facciale}
\paragraph{Pilastro anteriore} è la struttura più importante da individuare, tracciare e considerare in tutto il tracciato. Esso corrisponde ad una struttura anatomica ben definita dall'addensamento trabecolare di rinforzo in corrispondenza della branca anteriore del mascellare. Si tratta di una struttura paramediana ma unica nelle teleradiografie latero-laterali.\\

Il pilastro anteriore è espresso dalla linea \punto{CF1} tracciata da \punto{Fm}, cioè un millimetro e mezzo dietro \punto{M} in corrispondenza della perpendicolare a \punto{C3}, tracciata dalla cresta di rinforzo sul fondo del seno frontale (talvolta sono individuabili due creste e la perpendicolare a C3 si traccia a partire dal punto di mezzo tra queste).\\

Lungo il suo percorso, la linea CF1 incontra il punto orbitale inferiore e il canale palatino anteriore, corrispondente quasi sempre all'apice del canino permanente o all'hipomoclion del canino in formazione. XXX FIXME XXX

Questi tre punti, soprattutto nelle terze classi, non sono ben allineati e il pilastro anteriore si fa passare per il punto \punto{Fm} e \punto{orbitale inferiore}. Il pilastro anteriore è quello che subisce maggiori alterazioni nelle varie patologie seguendo le alterazioni del mascellare superiore, che può avere un aggetto verso l'esterno normale, aumentato o diminuito. Esistono dunque due pilastri anteriori: quello \emph{reale}, dato dalle strutture che si considerano sulla teleradiografia, e quello \emph{ideale}, che considera lo sviluppo che il mascellare dovrebbe avere se tutte le strutture fossero nella norma. Dal confronto si ha la conoscenza della deviazione nella patologia considerata.\\

Il calcolo dell'aggetto del pilastro anteriore ideale prende in considerazione il fatto che la sua crescita in avanti aumenta con l'età e il suo valore angolare, calcolato tra \punto{C1} e \punto{CF1}, va da 80° nei bambini più piccoli a 90° negli adulti, con valori leggermente inferiori nelle donne. L'angolo ideale sarà tracciato considerando l'età del paziente, o anche aggiungendo al valore dell'angolo della base anteriore del cranio il valore di 65. Inoltre il pilastro ideale è tangente all'apice dell'incisivo inferiore, e taglia la sinfisi tra il terzo medio e quello posteriore.\\

Il primo dato diagnostico ci è riferito dal \emph{confronto tra pilastro reale e ideale}; il secondo, più importante, è dato dal \emph{punto di tangenza del pilastro ideale e la sinfisi}. Se cade sul bordo posteriore della sinfisi si può dire che la mandibola è grande; se cade davanti, che è piccola. Questi valori non devono però essere solo correlati tra loro, ma devono essere valutati altri dati -- per esempio, la rotazione della mandibola -- perché in caso di rotazione posteriore, una mandibola più grande della norma può esere tagliata da CF1 tra terzo medio e terzo posteriore della sinfisi o nella sua parte anteriore.\\

Altra situazione da considerare, sempre in relazione a CF1, è una mandibola piccola ma mesializzata, che si troverà in posizione di normalità rispetto a CF1. Questo dato sarà controllato dalla posizione del bordo posteriore della mandibola rispetto al bordo anteriore del tratto cervicale.

\paragraph{Pilastro medio} o \punto{CF2}, è il segmento che congiunge il \punto{bregma} al punto \punto{Pts}. Pts è il punto di mezzo del tratto postero-superiore della fessura pterigo-mascellare che si ottiene tracciando la bisettrice dell'angolo formato da \punto{C1} e la retta tangente al processo pterigoideo dello sfenoide.\\
La parte terminale di \punto{CF2} divide la mandibola in due parti uguali. Se le due parti non sono uguali ci si può trovare in presenza di una mandibola piccola o distalizzata (parte anteriore diminuita o postruotata). Se la parte posteriore è più piccola, la mandibola può essere in protrusione oppure aumentata di volume.\\

\paragraph{Pilastro posteriore} o \punto{CF3}, si traccia parallelamente a \punto{CF2} perché così è orientato nelle strutture normali, partendo dal punto di tangenza di \punto{C1} con il condilo, cioè il punto \punto{Cp}.

Anche in questo caso si tracciano due pilastri posteriori, uno ideale, parallelo, e uno reale, tangente al bordo posteriore della mandibola, sempre partendo dal punto \punto{Cp}. Nei casi di normalità di posizione mandibolare i due coincidono, nel caso contrario possono indicare una mandibola retrusa o protrusa, eccessi di sviluppo della zona dell'angolo mandibolare o un'eccessiva inclinazione della branca montante con apertura dell'angolo.

\paragraph{Piano bispinale} o \punto{CF4}. Nella norma, la spina nasale posteriore e quella anteriore sono allineate con l'articolazione occipito-rachidea, il dente dell'epistrofeo e l'arco dell'atlante.\\

In un'età vicina alla pubertà il dente dell'epistrofeo e l'arco dell'atlante sono tagliati nella loro parte superiore, ma in un bambino di giovane età si trovano al di sotto della congiungente i condili occipitali e la linea bispinale. La ragione risiede nel fatto che nel processo di crescita le vertebre cervicali si sviluppano più verso l'alto che il basso. Questi punti possono quindi coincidere o meno. Quando nessun punto coincide con l'articolazione occipito-rachidea si devono tracciare diverse linee passanti per i diversi punti, cioè si costituisce una \emph{banda} e la linea di mezzo sarà quella alla quale ci si riferisce.\\
\\
Da CF4 risulta evidente la posizione della spina nasale anteriore, che può trovarsi verso l'alto, come nei morsi aperti, o verso il basso, nei morsi chiusi. La stessa sorte può subire la spina nasale posteriore che, a seconda dei movimenti della lingua, può orientarsi verso l'alto o verso il basso.

\paragraph{Altezza facciale} espressa da \punto{CF5}, è tracciata a partire dal punto SNA, reale se si colloca sul piano bispinale, oppure ideale se si trova sul CF4 ricavato dalla banda di linee di cui sopra. Da questo punto si traccia una perpendicolare fino al punto \punto{N}, o meglio \punto{Na'}, perché la retta giace al di fuori del cranio e, nei casi di ipomaxillia grave, talvolta all'interno.\\

\punto{CF5}, misurato come \piano{Na'}{Met'}, viene calcolato raddoppiando la distanza \piano{SNA}{Na'} e aggiungendo un nono della misura trovata. Il punto trovato è \punto{Met'}, e viene riportato sul pilastro ideale, tracciando una perpendicolare da \punto{Met'} al pilastro stesso, e viene indicato con \punto{Met}.\\

\paragraph{Piano mandibolare} \punto{CF6}, indica il piano mandibolare tracciato da \punto{Met} e tangente alla squama dell'occipitale. Incontra \punto{CF4} in un punto \punto{OM}. Questa linea indica lo sviluppo del bordo inferiore della mandibola, il grado dello sviluppo del ramo perlopiù diminuito nelle iperdivergenze, lungo per esempio in certe sindromi come l'acromegalia e in certi casi di \emph{long face} a ramo lungo. Evidenzierà anche il grado di post-rotazione mandibolare, il cui bordo eccederà dal piano mandibolare stesso.

\paragraph{Piano occlusale} espresso da \punto{CF7}, si presenta anch'esso in due modalità, una reale e una ideale. Il piano occlusale reale corrisponde al piano occlusale di Ricketts, mentre quello ideale corrisponde alla metà della distanza tra \piano{SNA}{Met'}, che si congiunge indietro nel punto \punto{OM}.\\

\paragraph{CF8} è l'ultimo piano da considerare, tracciato parallelo a \punto{C3} a partire da \punto{SNA}, reale o ideale, fino a raggiungere la zona dell'angolo che potrà coincidere o no col vertice dell'angolo. Dà utili indicazioni sull'angolo e sulla lunghezza del ramo della mandibola e la lunghezza di questo in relazione a quella della branca montante del mascellare. Nella norma, il ramo mandibolare è più lungo del mascellare.

%La ricchezza degli elementi considerati, e soprattutto il fatto che gli elementi scheletrici siano cranio-maxillo-rachidei, permette una grande elaborazione di dati: correlando gli uni agli altri si può, per via induttiva e deduttiva, porre le basi per una buona diagnosi.

\section{Analisi dentale}
