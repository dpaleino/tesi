\chapter{Crescita cranio-facciale}
\nocite{Enlow1986,Cozza2006}

Le teorie di crescita rappresentano le ipotesi postulate nel corso dell'evoluzione scientifica al fine di spiegare e identificare i fattori responsabili dello sviluppo del complesso cranio-facciale.

Per meglio utilizzare le tecniche di analisi cefalometrica più avanti descritte, l'or\-to\-don\-ti\-sta deve conoscere i comuni meccanismi di crescita, per sapere in quale momento è necessario intervenire, e capire le condizioni che possono favorire la stabilità del risultato terapeutico o, al contrario, causarne la recidiva.

L'analisi dell'evoluzione storica sull'argomento ha permesso di dividere le teorie di crescita in tre gruppi principali:
\begin{enumerate}
\item corrente di pensiero genetica (1931-1946)
\item corrente di pensiero funzionalistica (1945-1990)
\item corrente di pensiero sintetica o del consenso (1970)
\end{enumerate}

\section{Corrente di pensiero genetica (1931-1946)}
Le prime teorie sul processo di crescita ritenevano che il meccanismo principale era sotto un costante e rigoroso controllo genico. Tale condizione metteva fortemente in discussione l'efficacia dell'ortopedia dento-facciale in generale, e di quella funzionale in particolare.

La crescita del cranio è predeterminata, e non è soggetta ad alcuna influenza esterna, pertanto anche le dismorfosi dento-maxillo-facciali sono manifestazioni di caratteri ereditari.

\subsection*{Teoria della predeterminazione genetica\protect\footcite{Broadbent1931,Broadbent1937,Brodie1941,Brodie1946}}
La prima teoria postulata nel corso della letteratura affermava che le disgnazie esistono prima della nascita, codificate nei geni, e che durante la crescita post-natale non migliorano e non peggiorano, pertanto quando un neonato nasce con una disgnazia, questa permane tutta la vita, e non può essere trattata in alcun modo.

\subsection*{Ipotesi suturale\protect\footcite{Weinmann1956}}
Le suture, la cartilagine e il periostio sono ``centri autonomi'' di crescita sotto il controllo genico, e non influenzabili da fattori locali né dalla terapia. Ci si sposta quindi da una teoria in cui tutto è sotto il controllo genico, ad una in cui l'attività genica si limita a quei tessuti capaci di generare osso.

\subsection*{Ipotesi del setto nasale\protect\footcite{Scott1967}}
Tale ipotesi si basa sul ruolo svolto dalle strutture cartilaginee del cranio durante lo sviluppo fetale, e sulla possibilità che esse continuino il loro ruolo di guida dello sviluppo anche post-nascita.

Il controllo genico si sposta quindi sulle strutture cartilaginee, che diventano le uniche responsabili del processo di crescita cranio-facciale.

In particolare, la cartilagine del setto nasale è responsabile della crescita del mascellare superiore: con il suo sviluppo, si ha una spinta verso il basso e in avanti della premaxilla e, insieme alla \textit{cartilagine di Meckel}, partecipa in modo preponderante alla formazione della faccia.

La limitazione dell'influenza genica alle sole cartilagini apre quindi la strada ad una possibile influenza esterna sulle suture, sfruttabile in ambito terapeutico.

\section{Corrente di pensiero funzionalistica (1945-1990)}
La corrente funzionalistica mette in evidenza l'importanza della funzione nella realizzazione della crescita cranio-facciale, spostando l'attenzione dal pensiero genetico a quello funzionale.

L'influenza genetica è comunque presente in tutto lo sviluppo biologico; i tessuti hanno però un loro grado di plasticità e quindi sono influenzabili da fattori estrinseci al genoma.

Se la funzione rappresenta il fattore più importante per la crescita, sarà anche quello che alterandosi causerà una disgnazia; pertanto la riabilitazione della corretta funzione determina il recupero dell'equilibrio.

\subsection*{Ipotesi della matrice funzionale\protect\footcite{Moss1960,Moss1969}}
In questa teoria, è la funzione ad avere un'influenza diretta su forma, dimensione e posizione dei tessuti scheletrici; anche se esiste un controllo genico durante la fase iniziale della ossificazione, questo continua poi a livello funzionale. Le strutture ossee e cartilaginee non sono infatti dotate di un proprio schema di crescita, ma si accrescono secondariamente ai tessuti che li circondano (\textit{matrici funzionali}).

Secondo Moss, esistono delle funzioni vitali (tra cui masticazione, fonazione, deglutizione), ed ognuna di queste è svolta grazie a tessuti, organi, spazi e strutture scheletriche e cartilaginee. L'insieme delle entità anatomiche necessarie per eseguire una specifica funzione viene detta \textit{componente cranica funzionale}: ciascuna di queste è costituita da due elementi -- la \textit{matrice funzionale}, che svolge la funzione propria; e l'\textit{unità scheletrica} che svolge la funzione di protezione e di sostegno.

La grandezza, la forma e la posizione di ogni unità scheletrica rappresentano una risposta compensatoria alle richieste della matrice funzionale; l'unità scheletrica non è quindi direttamente regolata dal genoma, ma viene modulata dalla matrice funzionale.

Moss distingue due matrici funzionali:

\begin{enumerate}
\item la \textit{matrice funzionale periostale}, tipicamente associata a muscoli, vasi sanguigni, nervi e ghiandole;
\item la \textit{matrice funzionale capsulare}, costituita da capsule o involucri di tessuti non scheletrici che includono la loro unità scheletrica.
\end{enumerate}

La crescita a livello della matrice funzionale \textit{periostale} è di tipo \textit{trasformativo}, e si realizza attraverso processi di apposizione e riassorbimento osseo, che inducono una modificazione della forma e della dimensione della propria unità scheletrica. Le matrici di questo tipo hanno quindi influenza su \textit{microunità scheletriche} che, prese insieme, formano un intero osso.

Un esempio è la mandibola, che risulta costituita da 5 microunità: condilare, coronoidea, angolare, alveolare e basale. Il muscolo temporale è la matrice dell'unità coronoidea; il massetere e pterigoideo interno di quella angolare; i denti sull'unità alveolare, mentre il fascio vascolonervoso del canale mandibolare agisce sull'unità basale.

Per quanto riguarda le matrici funzionali capsulari, nel distretto cranio-facciale si riconoscono:

\begin{enumerate}
\item \textit{matrice funzionale capsulare neurocranica}, che rappresenta il volume della massa cerebrale;
\item \textit{matrice funzionale capsulare orofacciale}, di nostra pertinenza, che rappresenta il volume degli spazi funzionanti delle cavità oro-naso-faringee, orbitali e uditive.
\end{enumerate}

La crescita a livello delle matrici \textit{capsulari} è di tipo \textit{traslativo}, avviene cioè attraverso un processo di riposizionamento della propria unità scheletrica. La sfera d'influenza è rappresentata dalle \textit{macrounità scheletriche}. Ciascuna matrice funzionale capsulare, e ciascuna capsula, contiene quindi le matrici funzionali periostali e le rispettive microunità scheletriche.

Lo sviluppo di tutte le unità scheletriche cranio-facciali è quindi una combinazione di due tipologie di crescita:

\begin{enumerate}
\item \textit{trasformativa}, dovuta ai cambiamenti in forma e dimensione delle microunità scheletriche, in risposta allo stimolo delle matrici funzionali periostali;
\item \textit{traslativa}, dovuta ad un ricollocamento nello spazio delle macrounità scheletriche, in risposta all'aumento volumetrico degli spazi funzionanti e della massa cerebrale.
\end{enumerate}

L'insieme della crescita trasformativa e della crescita traslativa permette il mantenimento dell'equilibrio tra matrici funzionali e unità scheletriche, tale da realizzare una \textit{crescita armonica}.

\subsection*{Teoria del servosistema o teoria cibernetica\protect\footcite{Petrovic1974,Petrovic1981}}

Allo scopo di comprendere i meccanismi dello sviluppo craniofacciale, Petrovic e coll. hanno sviluppato la \textit{teoria del servosistema}, utilizzando il vocabolario proprio della cibernetica.

La cibernetica è una scienza che ha come obiettivo la comprensione sistematica della realtà, facendo confluire insieme una serie di nozioni e problemi patrimonio comune di più discipline (biologia, ingegneria, psicologia, meccanica). Tale scienza opera attraverso un circuito, il \textit{servosistema}, caratterizzato da un insieme di \textit{segnali di controllo} e \textit{di comando}, che schematizza come la trasmissione di tali segnali porta al verificarsi di un fenomeno.

La crescita delle differenti regioni del cranio diventa quindi la risultante dell'in\-te\-ra\-zio\-ne di un insieme di avvenimenti e di meccanismi di feedback, che permettono al sistema di autoregolarsi.

Petrovic ha quindi sintetizzato le sue idee nella teoria cibernetica dei processi di controllo della crescita cranio-facciale, attraverso la costruzione di un \textit{servosistema}.

In generale, all'interno di un servosistema esiste un \textit{comparatore periferico} che opera un confronto tra un \textit{input} (variabile indipendente) e un \textit{output} (variabile controllata), e invia un messaggio al \textit{comparatore centrale}, che regola i fattori di controllo di un determinato fenomeno per mantenere l'equilibrio. Nel servosistema ideato da Petrovic, il comparatore periferico è rappresentato dall'\textit{occlusione}, quello centrale è il \textit{sistema nervoso centrale}, l'input è il \textit{mascellare superiore}, e l'output è la \textit{mandibola}.

L'occlusione, quindi, opera un confronto tra la posizione del mascellare superiore e la posizione della mandibola, e invia un messaggio al sistema nervoso centrale, che regola i fattori di controllo della crescita per mantenere l'equilibrio.

Il mascellare superiore è considerato la variabile indipendente in quanto è controllato solamente da fattori estrinseci generali (ormoni), e non può in alcun modo essere influenzato da fattori locali. La mandibola, invece, oltre a subire il controllo ormonale, è anche sottoposta a fattori estrinseci locali (muscoli, legamenti).
