\chapter{Analisi proporzionale di Coben}
\nocite{Coben1979,Manetti1984,Coben1985,Antonini1986}

Quest'analisi inquadra il cranio in un sistema di coordinate rettangolare, in cui l'asse orizzontale delle ascisse è il \textit{Basion Orizzontale} (\punto{BaH}), il punto d'origine è il \textit{Basion} (\punto{Ba}), e l'asse delle ordinate è il \textit{Basion Verticale} (\punto{BaV}). Le misurazioni della profondità facciale sono l'espressione della componente orizzontale della crescita, relativa al \textit{forame occipitale} (o \textit{forame magno}). In maniera simile, le misurazioni dell'altezza facciale sono espressione della componente verticale di crescita, relativa al forame occipitale.

Le misurazioni orizzontali vengono prese parallelamente a \punto{BaH}, quelle verticali parallelamente a \punto{BaV}. Il sistema di coordinate rettangolare permette misure lineari dei segmenti craniofacciali e, attraverso un sistema di proporzioni, la valutazione dei rapporti tra i singoli segmenti, fornendo un profilo individuale.

Quest'analisi si compone di tre parti: l'indice di profondità, la profondità e l'equilibrio verticale. Non vengono utilizzati punti cefalometrici particolari, a parte \textit{Incision superiore (Is)} e \textit{Incision inferiore (Ii)}, già descritti nell'analisi di Ricketts.

\section{Descrizione}
\subsection{Valutazione dell'indice di profondità}
È possibile definire \textit{indice di profondità} il rapporto tra l'altezza facciale anteriore (\piano{Na}{Me}) e la profondità facciale (\piano{Ba}{Na}). I valori medio-normali di questo rapporto, utile nel fornire indicazioni sull'armonia dello sviluppo facciale, sono di $115,5 \pm 6,56$ in fase di dentatura mista e $123,9 \pm 4,85$ in dentatura permanente.

\subsection{Analisi della profondità}
Per descrivere l'analisi della profondità facciale, è utile suddividerla in tre livelli:
\begin{itemize}
\item livello ``base cranica''
\item livello ``mascellare superiore''
\item livello ``mandibola''
\end{itemize}

Nel livello ``\textit{base cranica}'' vengono considerate le seguenti misurazioni:

\begin{description}
\item[\piano{Ba}{Na}] è la profondità totale della faccia, e una delle poche misure assolute di quest'analisi. È propria di ciascun soggetto, e ad essa vengono rapportate tutte le misure antero-posteriori. Il suo valore medio è di $89,9 \pm 2,1$mm.
\item[\piano{Ba}{S}/\piano{Ba}{Na}] profondità del basisfenoide o, in altri termini, l'inclinazione della fossa cranica media. Il suo valore medio è di $25,4 \pm 1,4$\%.
\end{description}

A livello del \textit{mascellare superiore}, vengono considerate:

\begin{description}
\item[\piano{Ba}{A}/\piano{Ba}{Na}] profondità del mascellare superiore, valore medio $97,7 \pm 1,8$\%.
\item[\piano{S}{SNP}/\piano{Ba}{Na}] localizzazione antero-posteriore del mascellare superiore, valore medio $20,2 \pm 1,2$\%.
\item[\piano{SNP}{A}/\piano{Ba}{Na}] dimensione sagittale del mascellare superiore, valore medio $52,1 \pm 1,8$\%.
\end{description}

A livello \textit{mandibolare}, si misurano:

\begin{description}
\item[\piano{Ba}{Po}/\piano{Ba}{Na}] profondità della mandibola, valore medio $90,1 \pm 6,38$\%.
\item[\piano{Ba}{Ar}/\piano{Ba}{Na}] localizzazione antero-posteriore della mandibola, valore medio $8,8 \pm 1,63$\%.
\item[\piano{Ar}{Go}/\piano{Ba}{Na}] inclinazione posteriore del ramo, valore medio $7,8 \pm 1,77$\%.
\item[\piano{Go}{Po}/\piano{Ba}{Na}] profondità del corpo della mandibola, valore medio $79,1 \pm 4,2$\%.
\end{description}

\subsection{Analisi dell'equilibrio verticale}
Si distinguono un livello anteriore ed un livello posteriore. Per quanto riguarda il primo, si considerano:

\begin{description}
\item[\piano{Na}{Me}] è, come il \piano{Ba}{Na}, un valore assoluto individuale, a cui si rapportano tutte le misure individuali. Rappresenta l'altezza anteriore della faccia, ed ha un valore medio di $123,9 \pm 4,85$mm.
\item[\piano{Na}{SNA}/\piano{Na}{Me}] è l'altezza del mascellare, ha un valore medio di $46 \pm 2,18$\%.
\item[\piano{SNA}{Me}/\piano{Na}{Me}] è l'altezza inferiore della faccia, ha un valore medio di $54 \pm 2,18$\%.
\item[\piano{SNA}{Is}/\piano{Na}{Me}] è l'altezza del processo alveolare del mascellare superiore, ha un valore medio di $23,2 \pm 1,58$\%.
\item[\piano{Ii}{Me}/\piano{Na}{Me}] è l'altezza del processo alveolare e della sinfisi mentoniera a livello mandibolare, ha un valore medio di $34,1 \pm 1,68$\%.
\end{description}

Per quanto riguarda il livello posteriore, si considerano:

\begin{description}
\item[\piano{S}{Go}/\piano{Na}{Me}] rappresenta l'altezza totale posteriore della faccia, rapportata con l'altezza totale anteriore. Ha un valore medio di $68,8 \pm 2,4$\%.
\item[\piano{S}{Ar}/\piano{Na}{Me}] rappresenta la posizione del condilo rispetto alla sella su un piano verticale, ha un valore medio di $26,5 \pm 1,84$\%.
\item[\piano{Ar}{Go}/\piano{Na}{Me}] rappresenta l'altezza della branca ascendente, ha un valore medio di $42,3 \pm 2,41$\%.
\end{description}

\section{Considerazioni}
