\footnotesize
\begin{longtable}{>{\bfseries}p{5cm}X}
Articolare di Björk \newline Ar & Punto bilaterale di intersezione tra il bordo inferiore del massiccio sfeno-occipitale e la superficie posteriore dei condili. Non rappresenta una struttura ossea ma un'immagine radiografica di sovrapposizione tra due strutture ossee: si identifica con l'articolazione temporomandibolare dove il condilo emerge dalla cavità glenoidea.\\\\
Basion \newline Ba & Punto più basso sul margine anteriore del forame magno, alla base del \textit{clivus}.\\\\
Bolton \newline Bo & Punto più alto della curvatura tra il condilo occipitale e la parte basilare dell'osso omonimo, localizzato al di dietro del condilo occipitale.\\\\
Bregma \newline Br & Punto sulla calotta cranica nel quale si incontrano la sutura coronale e la sutura sagittale.\\\\
Contatto molare \newline Cm & Punto bilaterale di contatto molare disto-occlusale delle cuspidi dei primi molari permanenti.\\\\
Contatto molare deciduo \newline cm & Punto bilaterale di contatto molare disto-occlusale delle cuspidi dei secondi molari decidui.\\\\
Fessura pterigo-maxillare \newline Ptm & Punto alla base della fessura pterigo-maxillare, in cui si incontrano le pareti anteriore e posteriore.\\\\
Fondo della sella \newline IST & Punto più inferiore del contorno della sella turcica.\\\\
Fronto-mascellare \newline FMS & Punto posteriore della sutura fronto-maxillare.\\\\
Menton \newline Me & Punto d'incontro tra la sinfisi mentoniera e il corpo mandibolare.\\\\
Nasion \newline Na & Punto di intersezione anteriore tra l'osso frontale e quello nasale.\\\\
Gnathion \newline Gn & Centro del contorno inferiore del mento.\\\\
Gonion \newline Go & Punto bilaterale di costruzione geometrica dall'intersezione di due rette: una passante dal Menton al punto più basso del margine postero-inferiore della mandibola, l'altra dal punto Articolare al punto più sporgente del margine posteriore del ramo della mandibola.\\\\
Processo clinoideo \newline Clp & Apice del processo clinoideo della sella turcica.\\\\
Prosthion inferiore \newline IPr & Punto mediano più sporgente del processo alveolare della mandibola, tra gli incisivi centrali inferiori.\\\\
Prosthion superiore \newline SPr & Punto mediano più sporgente del processo alveolare del mascellare, tra gli incisivi centrali superiori.\\\\
Pterigoideo \newline Pts & Punto più superiore della fessura pterigo-maxillare.\\\\
Sella \newline S & Centro geometrico della sella turcica.\\\\
Sfeno-etmoidale \newline Se & Punto d'intersezione tra la base cranica anteriore e le grandi ali dello sfenoide.\\\\
Sopramentale di Downs \newline B & Punto mediano più retruso della concavità anteriore della mandibola, tra il processo alveolare e la prominenza sinfisaria anteriore.\\\\
Sottospinale di Downs \newline A & Punto mediano più retruso della concavità anteriore del mascellare, tra la spina nasale anteriore e il processo alveolare.\\\\
Sopraorbitario \newline SOr & Punto d'incontro del tetto dell'orbita con il margine esterno dell'orbita stessa.\\\\
Spina nasale anteriore \newline SNA & Punto più anteriore della spina nasale anteriore.\\\\
Spina nasale posteriore \newline SNP & Punto più posteriore sul palato osseo.\\\\
Temporo-condilare \newline CT & Punto più inferiore della superficie postero-inferiore del tubercolo articolare dell'osso temporale.\\\\
Tuberosità linguale \newline LT & Punto bilaterale ottenuto dall'intersezione del piano occlusale con il margine anteriore del ramo mandibolare; rappresenta l'abitacolo per l'ultimo molare (primo, secondo o terzo molare a seconda della fase di maturazione della dentatura).\\\\
\end{longtable}
\normalsize
