\begin{longtable}{>{\bfseries}p{5cm}X}
Fronto-mascellare \newline FM & Punto mediano di unione tra osso frontale, osso mascellare e osso nasale.\\\\
Sutura sfeno-etmoidale \newline Se & Punto mediano di intersezione del profilo anteriore della cresta sfenoidale con il pavimento della fossa cranica media.\\\\
Fessura pterigo-mascellare \newline PTM & Punto più basso della fessura pterigo-mascellare (area bilaterale radiotrasparente a forma di goccia).\\\\
Articolare di Björk \newline Ar & Punto bilaterale di intersezione tra il bordo inferiore del massiccio sfeno-occipitale e la superficie posteriore dei condili. Non rappresenta una struttura ossea ma un'immagine radiografica di sovrapposizione tra due strutture ossee: si identifica con l'articolazione temporomandibolare dove il condilo emerge dalla cavità glenoidea.\\\\
Contatto molare \newline Cm & Punto bilaterale di contatto molare disto-occlusale delle cuspidi dei primi molari permanenti.\\\\
Contatto molare deciduo \newline cm & Punto bilaterale di contatto molare disto-occlusale delle cuspidi dei secondi molari decidui.\\\\
Sottospinale di Downs \newline A & Punto mediano più retruso della concavità anteriore del mascellare, tra la spina nasale anteriore e il processo alveolare.\\\\
Prosthion superiore \newline SPr & Punto mediano più sporgente del processo alveolare del mascellare, tra gli incisivi centrali superiori.\\\\
Sopramentale di Downs \newline B & Punto mediano più retruso della concavità anteriore della mandibola, tra il processo alveolare e la prominenza sinfisaria anteriore.\\\\
Prosthion inferiore \newline IPr & Punto mediano più sporgente del processo alveolare della mandibola, tra gli incisivi centrali inferiori.\\\\
Menton \newline Me & Punto mediano più basso situato sulla curva inferiore della sinfisi.\\\\
Gonion \newline Go & Punto bilaterale di costruzione geometrica dall'intersezione di due rette: una passante dal Menton al punto più basso del margine postero-inferiore della mandibola, l'altra dal punto Articolare al punto più sporgente del margine posteriore del ramo della mandibola.\\\\
Tuberosità linguale \newline LT & Punto bilaterale ottenuto dall'intersezione del piano occlusale con il margine anteriore del ramo mandibolare; rappresenta l'abitacolo per l'ultimo molare (primo, secondo o terzo molare a seconda della fase di maturazione della dentatura).\\\\
Sella \newline S & Centro geometrico della sella turcica.\\\\
Basion \newline Ba & Punto più basso e posteriore dell'osso occipitale.\\\\
Pogonion \newline Pg & Punto più anteriore della sinfisi mentoniera.\\\\
Spina nasale anteriore \newline SNA & Punto più anteriore della spina nasale anteriore.\\\\
Spina nasale posteriore \newline SNP & Punto più posteriore sul palato osseo.\\\\

\end{longtable}