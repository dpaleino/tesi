\begingroup
\let\clearpage\relax
\let\cleardoublepage\relax
\let\cleardoublepage\relax

\null\vspace*{\stretch{0.8}}

%\appendix
\chapter{Radiografie e tracciati}

\null\vspace*{\stretch{1}}

\endgroup

%\chapter{Radiografie e tracciati}
%\footnotetext{Pagina intenzionalmente vuota.}
\clearpage
\paziente{1}{VAFE1997}{F}{13}
\paziente{2}{DABA1994}{M}{16}
\paziente{3}{MASCHI2000}{M}{10}
\paziente{4}{ALELO1994}{M}{16}
\paziente{5}{EMALO2001}{M}{8}
\paziente{6}{MAMI2000}{F}{9}
\paziente{7}{AHMO1996}{M}{9}
\paziente{8}{RONA1997}{F}{12}
\paziente{9}{CAQUA1998}{M}{12}
\paziente{10}{MAZO1997}{F}{13}
\paziente{11}{STEZO2001}{F}{9}
\paziente{12}{SAGAFE2001}{M}{7}
\paziente{13}{SATO2002}{M}{9}
\paziente{14}{ALERA1999}{M}{11}
\paziente{15}{ALEPO1997}{F}{13}
\paziente{16}{SEMA2001}{F}{9}
\paziente{17}{SOMA1996}{F}{14}
\paziente{18}{TILO1999}{F}{12}
\paziente{19}{ELITRI1998}{F}{13}
%Questa paziente presenta un leggero squilibrio verticale, con la componente verticale posteriore più piccola di 7 mm rispetto a quella anteriore. Esiste anche uno squilibrio orizzontale posteriore tra la base cranica e il ramo mandibolare, più grande di 11 mm rispetto alla prima. A livello anteriore, esiste uno squilibrio a livello basale, con la controparte mandibolare basale più grande di 4 mm rispetto alla controparte mascellare, denotando una III classe scheletrica. L'angolo della base cranica è di 39°, e questo spiega la protrusione mandibolare di compenso, con una rotazione antioraria dell'angolo goniaco e uno scarso sviluppo nasomascellare.

%Rispetto ad Enlow, non viene rilevata la III classe (l'\angolo{ANB} è pari a 3°). È presente una post-rotazione del piano occlusale, indicata dal valore dell'angolo cranio-occlusale superiore alla media. L'elevato angolo intermascellare denota inoltre iperdivergenza, accompagnata da ipermaxillia (dimensione sagittale mascellare pari a 48 mm), e ipoplasia verticale del ramo mandibolare. A livello anteriore, Giannì denota inoltre un deep-bite scheletrico.

%A livello dentario, viene indicata una vestibolarizzazione degli incisivi, sia superiori che inferiori. Nel caso dei superiori, è errata l'angolazione rispetto al piano occlusale, e la posizione rispetto al piano \piano{N}{A}. Nel caso degli inferiori, è errata l'inclinazione e la posizione, entrambi rispetto al piano \piano{N}{B}. Anche l'angolo interincisale è minore della media, indicando una vestibolarizzazione degli incisivi considerati nel loro insieme.

\paziente{20}{SHATRI1996}{F}{15}
\paziente{21}{SALA1993}{M}{17}
\paziente{22}{FLOCUR1996}{F}{14}
\paziente{23}{MARVOL1993}{F}{13}
\paziente{24}{MAUZ1996}{F}{8}
\paziente{25}{FARI1988}{F}{20}
\paziente{26}{CHIPIN1996}{F}{11}
\paziente{27}{GAPO2001}{F}{9}
\paziente{28}{ALELO1997}{F}{13}
\paziente{29}{CALU2002}{F}{9}